\documentclass[a4paper,11pt,twoside,openright]{report}
% KOMPILOWAĆ ZA POMOCĄ pdfLaTeXa, PRZEZ XeLaTeXa MOŻE NIE BYĆ POLSKICH ZNAKÓW

% -------------- Kodowanie znaków, język polski -------------

\usepackage[utf8]{inputenc}
\usepackage[MeX]{polski}
\usepackage[T1]{fontenc}
\usepackage[english,polish]{babel}


\usepackage{amsmath, amsfonts, amsthm, latexsym} % głównie symbole matematyczne, środowiska twierdzeń

\usepackage[final]{pdfpages} % inputowanie pdfa
%\usepackage[backend=bibtex, style=verbose-trad2]{biblatex}


% ---------------- Marginesy, akapity, interlinia ------------------

\usepackage[inner=20mm, outer=20mm, bindingoffset=10mm, top=25mm, bottom=25mm]{geometry}


\linespread{1.5}
\allowdisplaybreaks

\usepackage{indentfirst} % opcjonalnie; pierwszy akapit z~wcięciem
\setlength{\parindent}{5mm}


%--------------------------- ŻYWA PAGINA ------------------------

\usepackage{fancyhdr}
\pagestyle{fancy}
\fancyhf{}
% numery stron: lewa do lewego, prawa do prawego 
\fancyfoot[LE,RO]{\thepage} 
% prawa pagina: zawartość \rightmark do lewego, wewnętrznego (marginesu) 
\fancyhead[LO]{\sc \nouppercase{\rightmark}}
% lewa pagina: zawartość \leftmark do prawego, wewnętrznego (marginesu) 
\fancyhead[RE]{\sc \leftmark}

\renewcommand{\chaptermark}[1]{
\markboth{\thechapter.\ #1}{}}

% kreski oddzielające paginy (górną i dolną):
\renewcommand{\headrulewidth}{0 pt} % 0 - nie ma, 0.5 - jest linia


\fancypagestyle{plain}{% to definiuje wygląd pierwszej strony nowego rozdziału - obecnie tylko numeracja
  \fancyhf{}%
  \fancyfoot[LE,RO]{\thepage}%
  
  \renewcommand{\headrulewidth}{0pt}% Line at the header invisible
  \renewcommand{\footrulewidth}{0.0pt}
}



% ---------------- Nagłówki rozdziałów ---------------------

\usepackage{titlesec}
\titleformat{\chapter}%[display]
  {\normalfont\Large \bfseries}
  {\thechapter.}{1ex}{\Large}

\titleformat{\section}
  {\normalfont\large\bfseries}
  {\thesection.}{1ex}{}
\titlespacing{\section}{0pt}{30pt}{20pt} 
%\titlespacing{\co}{akapit}{ile przed}{ile po} 
    
\titleformat{\subsection}
  {\normalfont \bfseries}
  {\thesubsection.}{1ex}{}


% ----------------------- Spis treści ---------------------------
\def\cleardoublepage{\clearpage\if@twoside
\ifodd\c@page\else\hbox{}\thispagestyle{empty}\newpage
\if@twocolumn\hbox{}\newpage\fi\fi\fi}


% kropki dla chapterów
\usepackage{etoolbox}
\makeatletter
\patchcmd{\l@chapter}
  {\hfil}
  {\leaders\hbox{\normalfont$\m@th\mkern \@dotsep mu\hbox{.}\mkern \@dotsep mu$}\hfill}
  {}{}
\makeatother

\usepackage{titletoc}
\makeatletter
\titlecontents{chapter}% <section-type>
  [0pt]% <left>
  {}% <above-code>
  {\bfseries \thecontentslabel.\quad}% <numbered-entry-format>
  {\bfseries}% <numberless-entry-format>
  {\bfseries\leaders\hbox{\normalfont$\m@th\mkern \@dotsep mu\hbox{.}\mkern \@dotsep mu$}\hfill\contentspage}% <filler-page-format>

\titlecontents{section}
  [1em]
  {}
  {\thecontentslabel.\quad}
  {}
  {\leaders\hbox{\normalfont$\m@th\mkern \@dotsep mu\hbox{.}\mkern \@dotsep mu$}\hfill\contentspage}

\titlecontents{subsection}
  [2em]
  {}
  {\thecontentslabel.\quad}
  {}
  {\leaders\hbox{\normalfont$\m@th\mkern \@dotsep mu\hbox{.}\mkern \@dotsep mu$}\hfill\contentspage}
\makeatother



% ---------------------- Spisy tabel i obrazów ----------------------

\renewcommand*{\thetable}{\arabic{chapter}.\arabic{table}}
\renewcommand*{\thefigure}{\arabic{chapter}.\arabic{figure}}
%\let\c@table\c@figure % jeśli włączone, numeruje tabele i obrazy razem


% --------------------- Definicje, twierdzenia etc. ---------------


\makeatletter
\newtheoremstyle{definition}%    % Name
{3ex}%                          % Space above
{3ex}%                          % Space below
{\upshape}%                      % Body font
{}%                              % Indent amount
{\bfseries}%                     % Theorem head font
{.}%                             % Punctuation after theorem head
{.5em}%                            % Space after theorem head, ' ', or \newline
{\thmname{#1}\thmnumber{ #2}\thmnote{ (#3)}}%  % Theorem head spec (can be left empty, meaning `normal')
\makeatother

% ----------------------------- POLSKI --------------------------------

\theoremstyle{definition}
\newtheorem{theorem}{Twierdzenie}[chapter]
\newtheorem{lemma}[theorem]{Lemat}
\newtheorem{example}[theorem]{Przykład}
\newtheorem{proposition}[theorem]{Stwierdzenie}
\newtheorem{corollary}[theorem]{Wniosek}
\newtheorem{definition}[theorem]{Definicja}
\newtheorem{remark}[theorem]{Uwaga}



% ----------------------------- Dowód -----------------------------

%\makeatletter
%\renewenvironment{proof}[1][\proofname]
%{\par
%  \vspace{-12pt}% remove the space after the theorem
%  \pushQED{\qed}%
%  \normalfont
%  \topsep0pt \partopsep0pt % no space before
%  \trivlist
%  \item[\hskip\labelsep
%        \sc
%    #1\@addpunct{:}]\ignorespaces
%}
%{%
%  \popQED\endtrivlist\@endpefalse
%  \addvspace{20pt} % some space after
%}
%
%\renewcommand{\qedhere}{\hfill \qedsymbol}
%\makeatother





% -------------------------- POCZĄTEK --------------------------


% --------------------- Ustawienia użytkownika ------------------

\newcommand{\tytul}{Komputerowe wspomaganie detekcji obszaru kortów badmintona z~wykorzystaniem sieci neuronowych}
\renewcommand{\title}{Detecting badminton courts using computer vision powered \mbox{by neural networks}}
\newcommand{\type}{magisters} % inżyniers, magisters, licencjac
\newcommand{\supervisor}{\mbox{dr inż. Magdalena Jasionowska}}

% --------------------- Ustawienia Przemka ------------------
\usepackage{import}
\usepackage{float}
\usepackage[hidelinks]{hyperref}

% \usepackage[disable]{todonotes} % notes not showed
\usepackage[draft]{todonotes}
\newcommand{\TODO}[1]{
  \todo[inline, color=lightgray]{#1}
} % notes showed

% \newcommand{\comment}[1]{}  % comment not showed
\newcommand{\comment}[1]{
  \par {\bfseries \color{blue} #1 \par}
} %comment showed

\newcommand*{\myref}[1]{\hyperref[{#1}]{Patrz punkt \ref*{#1}, \textbf{\nameref*{#1}}}}
\newcommand*{\myrefx}[1]{\hyperref[{#1}]{\textbf{\nameref*{#1}}, str. \pageref*{#1}}}
\newcommand*{\myfigref}[1]{\hyperref[{#1}]{\textbf{}\ref*{#1}}}
\newcommand*{\patrz}[1]{\hyperref[{#1}]{(patrz rozdz. \ref*{#1} str. \pageref*{#1})}}


% http://tug.ctan.org/tex-archive/macros/latex/contrib/fancyhdr/fancyhdr.pdf
\makeatletter
\def\cleardoublepage{\clearpage\if@twoside \ifodd\c@page\else
\null\newpage
\if@twocolumn\hbox{}\newpage\fi\fi\fi}
\makeatother

\makeatletter
\newcommand*{\currentname}{\@currentlabelname}
\makeatother


% --------------------- Document ------------------

\begin{document}
\sloppy

\includepdf[pages=-]{title_page/strona_tytulowa-jeden-autor}
% \documentclass[12pt,twoside,a4paper]{article}
% ten dokument należy kompilować za pomocą XeLaTeX-a

\usepackage{fontspec}
\defaultfontfeatures{Ligatures=TeX}


\usepackage{graphicx}
\usepackage[vmargin = 25mm, hmargin = 20mm, bindingoffset=10mm]{geometry}


\newfontfamily\arial{arial.ttf}

\linespread{1.5}
\setlength{\parindent}{0mm}

% ---------------------- Do wypełnienia ------------------------------

\newcommand{\discipline}{Informatyka i Systemy Informacyjne}
\newcommand{\spec}{Metody Sztucznej Inteligencji} % jak nie ma, to skasować
\renewcommand{\title}{Komputerowe wspomaganie detekcji obszaru kortów badmintona z~wykorzystaniem sieci neuronowych}
\renewcommand{\author}{Przemysław Rząd}
\newcommand{\album}{264209}
\newcommand{\supervisor}{dr inż. Magdalena Jasionowska-Skop}
\renewcommand{\year}{2019}

% ------------------------------------------------------------------------------


% --------------- TU SIĘ ZACZYNA DOKUMENT -------------------------------------
% DO WYPEŁNIENIA JEST TYP PRACY (LIC, MGR, INŻ) ORAZ BYĆ MOŻE USUNIĘCIE LINIJKI ZE SPECJALIZACJĄ, RESZTY LEPIEJ NIE RUSZAĆ, A JAK RUSZAĆ, TO Z POSZANOWANIEM ZARZĄDZENIA REKTORA: https://www.bip.pw.edu.pl/var/pw/storage/original/application/fb48a514799968a2fe7fdd17d3bf3cbe.pdf

\begin{document}
\pagestyle{empty}

\begin{center}
\includegraphics[scale=1.]{img/politechnika} 
\vspace{70pt}


\includegraphics[scale=1.]{img/praca_mgr} % praca_lic LUB praca_mgr LUB praca_inz

{ \arial na kierunku \discipline
\\ w specjalności \spec % JEŚLI NIE MA, TO WYWALIĆ LUB ZAKOMENTOWAĆ

\vspace{40pt}
{\arial \large \title}

\vspace{50pt}

{\arial \huge \author}

\vspace{5pt}

Numer albumu \album

\vspace{40pt}

promotor \\
{\arial \supervisor}

\vspace{15pt}
 
%konsultacje (opcjonalnie)\\
%{\arial
%dr hab. K. Konsultant
%}

 \vfill
WARSZAWA \year \\
}
\end{center}

\end{document}



% ------------------ STRONA Z PODPISAMI AUTORA/AUTORÓW I PROMOTORA ------------------


\thispagestyle{empty}\newpage
\null

\vfill

\begin{center}
\begin{tabular}[t]{ccc}

............................................. & \hspace*{100pt} & .............................................\\
podpis promotora & \hspace*{100pt} & podpis autora


\end{tabular}
\end{center}



% ---------------------------- ABSTRAKTY -----------------------------
% W PRACY PO POLSKU, NAPIERW STRESZCZENIE PL, POTEM ABSTRACT EN

{
\begin{abstract}

\begin{center}
\tytul
\end{center}

\import{meta/}{streszczenie}

\end{abstract}
}

\null\thispagestyle{empty}\newpage

{\selectlanguage{english}
\begin{abstract}

\begin{center}
\title
\end{center}

\import{meta/}{abstract}

\end{abstract}
}


% --------------------- OŚWIADCZENIE -----------------------------------------


\null\thispagestyle{empty}\newpage

\null \hfill Warszawa, dnia ..................\\

\par\vspace{5cm}

\begin{center}
Oświadczenie
\end{center}

\indent Oświadczam, że pracę \type ką pod
tytułem ,,\tytul '', której promotorem jest \supervisor , wykonałam/wykonałem
samodzielnie, co poświadczam własnoręcznym podpisem.
\vspace{2cm}


\begin{flushright}
  \begin{minipage}{50mm}
    \begin{center}
      ..............................................

    \end{center}
  \end{minipage}
\end{flushright}

\thispagestyle{empty}
\newpage

\null\thispagestyle{empty}\newpage


% ------------------- 4. Spis treści ---------------------
\pagenumbering{gobble}
\tableofcontents
\thispagestyle{empty}

%\newpage % JEŻELI SPIS TREŚCI MA PARZYSTĄ LICZBĘ STRON, ZAKOMENTOWAĆ
% ALBO JAK KTOŚ WOLI WTEDY DWIE STRONY ODSTĘPU, DODAĆ \null\newpage

% -------------- 5. ZASADNICZA CZĘŚĆ PRACY --------------------
\null\thispagestyle{empty}\newpage
\pagestyle{fancy}
\pagenumbering{arabic}
\setcounter{page}{11} % JEŻELI Z POWODU DUŻEJ ILOŚCI STRON W SPISIE TREŚCI SIĘ NIE ZGADZA, TRZEBA ZMODYFIKOWAĆ RĘCZNIE

\import{chapters/}{index}

% -------------------- 6. Bibliografia -----------------------
% Bibliografia leksykograficznie wg nazwisk autorów
% Dla ambitnych - można skorzystać z~BibTeX-a

\import{meta/}{bibliografia}
\newpage
\thispagestyle{empty}
% \pagenumbering{gobble}



% --- 7. Wykaz symboli i skrótów - jeśli nie ma, zakomentować
%\chapter*{Wykaz symboli i skrótów}
%
%\begin{tabular}{cl}
%nzw. & nadzwyczajny \\
%* & operator gwiazdka \\
%$\widetilde{}$ & tylda
%\end{tabular}
%\\
%Jak nie występują, usunąć.
%\thispagestyle{empty}


% ----- 8. Spis rysunków - jeśli nie ma, zakomentować --------
\listoffigures
\addcontentsline{toc}{chapter}{\listfigurename}
\newpage
\thispagestyle{empty}
%Jak nie występują, usunąć.


% ------------ 9. Spis tabel - jak wyżej ------------------
%\renewcommand{\listtablename}{Spis tabel}
%\listoftables
%\thispagestyle{empty}
%Jak nie występują, usunąć.


% 10. Spis załączników - jak nie ma załączników, to zakomentować lub usunąć

\subimport{meta/dodatki/}{index}

%\chapter*{Spis załączników}
%\begin{enumerate}
%\item Załącznik 1
%\item Załącznik 2
%\item Jak nie występują, usunąć rozdział.
%\end{enumerate}
%\thispagestyle{empty}


\end{document}
