% !TeX root = ../../main.tex
\newpage
\section{Zaproponowana architektura sieci}
\label{sec:zaproponowana_architektura}

Ze względu na to, iż niewystarczająco dokładne wykrycie kortu może niekorzystnie wpływać na punktację meczy, oraz ze względu na fakt iż oprogramowanie może być uruchamiane na przenośnych komputerach, w ramach niniejszej pracy implementacja Mask R-CNN została zmodyfikowana w następujących krokach:

\begin{itemize}
	\item architektura sieci - została zmodyfikowana w celu osiągnięcia większej dokładności maski;
	\item hiperparametry sieci - zostały dobrane w celu zmniejszenia zapotrzebowania na pamięć GPU.
\end{itemize}

\TODO{Tu będzie trochę dokładniejszy opis}

\begin{algorithm}
  \subimport{mask_r-cnn/}{commoninput}
  \hspace*{\algorithmicindent} \verb|deconv_layers| - \deconvlayersdescription \\
  \subimport{mask_r-cnn/}{commonoutput}
  \begin{algorithmic}[1]
    \subimport{mask_r-cnn/}{firststepsofmask}
    \For {$\mathit{i} = 1$ to deconv\_layers }
      \State $\mathit{L} \gets [...L,\verb| deconvolution 2x2 with strides 2|]$
    \EndFor
    \subimport{mask_r-cnn/}{laststepsofmask}
	\end{algorithmic}
	\caption{Tworzenie podsieci maski w większej rozdzielczości}
	\label{alg:mask-r-cnn-modified}
\end{algorithm}
