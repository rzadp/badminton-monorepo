% !TeX root = ../../main.tex
\chapter{Podejście do problemu}

\subimport{mask_r-cnn/}{index}

\newpage
\section{Zaproponowana architektura sieci}
\label{sec:zaproponowana_architektura}

Ze względu na to, iż niewystarczająco dokładne wykrycie kortu może niekorzystnie wpływać na punktację meczy, oraz ze względu na fakt iż oprogramowanie może być uruchamiane na przenośnych komputerach, w ramach niniejszej pracy implementacja Mask R-CNN została zmodyfikowana w następujących krokach:

\begin{itemize}
	\item architektura sieci - została zmodyfikowana w celu osiągnięcia większej dokładności maski;
	\item hiperparametry sieci - zostały dobrane w celu zmniejszenia zapotrzebowania na pamięć GPU.
\end{itemize}

\TODO{Tu będzie dokładniejszy opis}

\subsection{Zbiór danych i generator sztucznych danych}
\label{sec:generator}

Problem skompletowania dostatecznie licznego zbioru treningowego został rozwiązany poprzez stworzenie generatora sztucznych obrazów.
Generator, zaimplementowany jako scena 3D w JavaScript pozwala na generowanie obrazów z~następującymi cechami:

\begin{itemize}
	\item zmienne kolory obiektów: podłogi, kortu, słupków;
	\item dynamiczne światło;
	\item zmienna pozycja zawodników;
	\item zmienne odległości między kortami.
\end{itemize}

Dzięki powyższym cechom możliwe jest przygotowanie sztucznych obrazów zbliżonych do obrazów z kamer przemysłowych stosowanych na meczach odbywających się w różnych lokalizacjach.

W celu, aby obraz stanowił rekord treningowy, wymagana jest tak zwana anotacja, czyli oznaczenie obszaru kortu. Do anotacji wykorzystano program VGG Image Annotator \cite{vgg-via}.
\\

\missingfigure{
  Tu będzie tabela przedstawiająca: \newline
  1. zbiór COCO \cite{coco}, ile klas, ile rekordów, jaki podział treningowy/walidacyjny \newline
  2. mój zbiór, bez sztucznych \newline
  3. mój zbiór, ze sztucznymi
}

\subsection{Uczenie transferowe}
\label{sec:uczenie-transferowe}

Uczenie transferowe polega na...
\TODO{Tu będzie dokładniejszy opis}

W pracy wykorzystano uczenie transferowe na dwa sposoby:

\begin{itemize}
	\item Wykorzystanie wytrenowanego na COCO modelu i przetransferowanie na tytułowy problem
	\item Generalizacja problemu na inne sporty
\end{itemize}
