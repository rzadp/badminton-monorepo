\chapter{Podejście do problemu}

\subimport{}{przeglad}

\subimport{mask_r-cnn/}{index}

\section{Zaproponowana architektura sieci}

W ramach niniejszej pracy implementacja Mask R-CNN została zmodyfikowana w następujących kwestiach:

\begin{itemize}
	\item Architektura sieci została zmodyfikowana w celu osiągnięcia większej dokładności maski (PROBLEM FALBANEK)
	\item Hiperparametry sieci zostały dobrane w celu zmniejszenia zapotrzebowania na pamięć GPU
\end{itemize}

(...) DOKŁADNIEJSZY OPIS

\subsection{Zbiór danych i generator}

Problem skompletowania dostatecznie licznego zbioru treningowego został rozwiązany poprzez stworzenie generatora sztucznych obrazków. Generator, zaimplementowany jako scena 3D w JavaScript, pozwala na generowanie obrazków z następującymi cechami:

\begin{itemize}
	\item Zmienne kolory podłogi, kortu, słupków
	\item Dynamiczne światło
	\item Zmienna pozycja zawodników
	\item Zmienne odległości między kortami
\end{itemize}

Dzięki powyższym cechom możliwe jest przygotowanie sztucznych obrazków udających zdjęcia meczy przeprowadzonych w różnych lokalizacjach.

Aby zdjęcie czy obrazek stanowiło rekord treningowy, wymagana jest tak zwana anotacja, czyli oznaczenie obszaru kortu.

Do anotacji wykorzystano VGG Image Annotator\footnote{http://www.robots.ox.ac.uk/~vgg/software/via/}.

(...) dokładniejszy opis końcowego zbioru, porównanie ze zbiorem COCO

\subsection{Uczenie transferowe}

Uczenie transferowe polega na... (...)

W pracy wykorzystano uczenie transferowe na dwa sposoby:

\begin{itemize}
	\item Wykorzystanie wytrenowanego na COCO modelu i przetransferowanie na tytułowy problem
	\item Generalizacja problemu na inne sporty
\end{itemize}

(...) DOKŁADNIEJSZY OPIS
