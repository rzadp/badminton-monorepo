% !TeX root = ../../main.tex
\newpage
\section{Uczenie transferowe}
\label{sec:uczenie-transferowe}

Uczenie transferowe polega na wykorzystaniu wiedzy zdobytej na potrzeby jednego problemu, i wykorzystanie jej na potrzeby rozwiązywania innego, ale powiązanego problemu.
Niniejsza praca korzysta z uczenia transferowego, wykorzystując sieć wytrenowaną na COCO i przystosowywując ją do tytułowej tematyki.

Rysunek \myfigref{fig:transferowe}, bazujący na Rysunku \myfigref{fig:mask_r_cnn} przedstawia implementację, dzięki której umożliwiono uczenie transferowe. Rozpoczynając trening od wag uzyskanych po treningu na zbiorze COCO, poszczególne fragmenty modelu zostały zamrożone i nie zostają aktualizowane podczas treningu. Dzięki temu czas treningu zostaje znacząco skrócony.

\begin{figure}[h]
  \centering
  \caption{Metoda uczenia transferowego w implementacji Mask R-CNN}
  \includegraphics[width=1\textwidth]{transferowe.png}
  \label{fig:transferowe}
\end{figure}
