% !TeX root = ../../main.tex
\newpage
\section{Uczenie transferowe}
\label{sec:uczenie-transferowe}

Uczenie transferowe polega na wykorzystaniu wiedzy zdobytej podczas rozwiązywania jednego problemu na potrzeby innego, powiązanego problemu.
Niniejsza praca korzysta z uczenia transferowego, wykorzystując sieć wytrenowaną na zbiorze COCO \cite{coco} i przystosowywując ją do badanej tematyki.

Rysunek \numberref{fig:transferowe} przedstawia zasadę działania uczenia transferowego, dzięki której umożliwiono uczenie transferowe. Rozpoczynając trening od wag uzyskanych po treningu na zbiorze COCO, poszczególne fragmenty modelu zostały zamrożone i nie zostają aktualizowane podczas treningu. Dzięki temu czas treningu zostaje znacząco skrócony.

\begin{figure}[h]
  \centering
  \includegraphics[width=1\textwidth]{transferowe.png}
  \caption{Zamrożone w ramach uczenia transferowego elementy w \textit{Mask R-CNN}}
  \label{fig:transferowe}
\end{figure}
