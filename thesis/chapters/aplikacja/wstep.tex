% !TeX root = ../../main.tex
\chapter{Aplikacja}
\label{sec:aplikacja}
W tym rozdziale zostanie omówiona dołączona do pracy aplikacja, która składa się z następujących elementów:
\begin{itemize}
  \item Generator sztucznych danych;
  \item Anotator danych;
  \item Zmodyfikowana implementacja sieci \textit{Mask R-CNN} \cite{matterport-mask-rcnn};
  \item Skrypt służący do trenowania sieci;
  \item Skrypt służący do obliczania skuteczności sieci, według miar opisanych w rozdziale \myrefx{sec:miary}
\end{itemize}

Wraz z instrukcją instalacji\footnote{\nameref{sec:instrukcja-instalacji}} oraz instrukcją użytkownika\footnote{\nameref{sec:instrukcja-uzytkownika}}, niniejszy rozdział umożliwia zreprodukowanie eksperymentów, których wyniki zawarto w niniejszej pracy.
Następujące podpunkty, które podsumowują wykonaną pracę, przedstawiają wykorzystanie poszczególnych elementów aplikacji na poszczególnych etapach pracy:

\begin{enumerate}
 \item Skonstruowanie zbiorów danych \textit{low} i \textit{high};
 \item Wygenerowanie sztucznych danych i skonstruowanie z nich dwóch nowych zbiorów na bazie zbiorów \textit{low} i \textit{high};
 \item Zaanotowanie obrazów z powyższych czterech zbiorów, dzięki czemu mogły zostać użyte do treningu sieci;
 \item Uruchomienie skryptu do trenowania, testując różne podziały na zbiory treningowe i walidacyjne\footnote{Patrz rozdział \myrefx{sec:podzial}}. Podział dający najlepsze wyniki wykorzystano w kolejnych eksperymentach;
 \item Uruchomienie skryptu do trenowania oryginalnej sieci \textit{Mask R-CNN} na czterech wymienionych wyżej zbiorach;
 \item Uruchomienie skryptu do trenowania zmodyfikowanej sieci \textit{Mask R-CNN} na czterech wymienionych wyżej zbiorach;
 \item Uruchomienie skryptu obliczającego wyniki przeprowadzonych treningów sieci.
\end{enumerate}



