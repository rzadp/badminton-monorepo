% !TeX root = ../../main.tex
\newpage
\section{Hiperparametry treningu}
\label{sec:hiperparametry}

Treningi przeprowadzone w ramach niniejszej pracy i opisane szczegółowo w~Rozdziale \numberref{sec:eksperymenty_wyniki} korzystają z~następującej strategii doboru współczynnika uczenia $LR$ (z ang. \textit{learning rate}):
\begin{itemize}
  \item wartość początkowa: $LR_{0} := 0.001$, według zaleceń autora implemetacji \cite{matterport-mask-rcnn};
  \item w~przypadku braku poprawy skuteczności sieci na zbiorze walidacyjnym w pięciu kolejnych epokach treningowych, mierzonej jako procent poprawnie zaklasyfikowanych pikseli w obrazie, parametr $LR$ zmniejsza się pięciokrotnie (ale nie mniej niż wartość minimalna $ LR_{max} = 0.0000001 $);
\end{itemize}

Próby wskazały na zwiększenie skuteczności sieci trenowanej z użyciem takiej strategii zmniejszania parametru.
