% !TeX root = ../../main.tex
\section{Podsumowanie}
Wraz z instrukcją instalacji (Dodatek A) oraz instrukcją użytkownika (Dodatek B), niniejszy rozdział umożliwia zreprodukowanie eksperymentów, których wyniki zawarto w niniejszej pracy.
Następujące podpunkty, które podsumowują wykonaną pracę, przedstawiają wykorzystanie poszczególnych elementów aplikacji na poszczególnych etapach pracy:

\begin{enumerate}
 \item Skonstruowanie zbiorów danych \textit{low} i \textit{high};
 \item Wygenerowanie sztucznych danych i skonstruowanie z nich dwóch nowych zbiorów na bazie zbiorów \textit{low} i \textit{high};
 \item Zaanotowanie obrazów z powyższych czterech zbiorów, dzięki czemu mogły zostać użyte do treningu sieci;
 \item Uruchomienie skryptu do trenowania, testując różne podziały na zbiory treningowe i walidacyjne. Podział dający najlepsze wyniki wykorzystano w kolejnych eksperymentach;
 \item Uruchomienie skryptu do trenowania oryginalnej sieci \textit{Mask R-CNN} na czterech wymienionych wyżej zbiorach;
 \item Uruchomienie skryptu do trenowania zmodyfikowanej sieci \textit{Mask R-CNN} na czterech wymienionych wyżej zbiorach;
 \item Uruchomienie skryptu obliczającego wyniki przeprowadzonych treningów sieci.
\end{enumerate}
