% !TeX root = ../../main.tex
\section{Podsumowanie}
Wraz z instrukcją instalacji (Dodatek A), niniejszy rozdział przybliża wykorzystywaną aplikację, która umożliwia zreprodukowanie eksperymentów, których wyniki zawarto w niniejszej pracy.
Następujące podpunkty, które podsumowują wykonaną pracę, przedstawiają wykorzystanie poszczególnych elementów aplikacji na poszczególnych etapach pracy:

\begin{enumerate}
 \item zebranie zbiorów danych \textit{low} i \textit{high};
 \item wygenerowanie sztucznych danych i skonstruowanie z nich dwóch nowych zbiorów na bazie zbiorów \textit{low} i \textit{high} - zbiór nazwany \textit{low+artificial} oraz zbiór nazwany \textit{high+artificial};
 \item zaanotowanie obrazów z powyższych czterech zbiorów, dzięki czemu mogły zostać użyte do treningu sieci;
 \item uruchomienie skryptu do trenowania, testując różne podziały na zbiory treningowe i walidacyjne (Rozdział \numberref{sec:podzial} oraz \numberref{sec:podzial_eksperyment}); podział dający najlepsze wyniki wykorzystano w kolejnych eksperymentach;
 \item uruchomienie skryptu do trenowania oryginalnej sieci \textit{Mask R-CNN} na czterech wymienionych wyżej zbiorach (\textit{low}, \textit{high}, \textit{low+artificial} oraz \textit{high+artificial});
 \item uruchomienie skryptu do trenowania zmodyfikowanej (Rozdział \numberref{sec:zaproponowana_architektura}) sieci \textit{Mask R-CNN} na czterech wymienionych wyżej zbiorach (\textit{low}, \textit{high}, \textit{low+artificial} oraz \textit{high+artificial});
 \item uruchomienie skryptu obliczającego wyniki przeprowadzonych treningów sieci.
\end{enumerate}
