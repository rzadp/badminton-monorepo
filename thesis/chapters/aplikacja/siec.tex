% !TeX root = ../../main.tex
\newpage
\section{Sieć \textit{Mask R-CNN}}

Sieć \textit{Mask R-CNN} \cite{matterport-mask-rcnn} jest otwartoźródłowym projektem wykorzystującym następujące technologie:
\begin{itemize}
  \item \textit{Python3}\footnote{https://www.python.org/}
  \item \textit{TensorFlow}\footnote{https://www.tensorflow.org/}
  \item \textit{Keras} \cite{keras}
\end{itemize}

Konkretne wersje i dodatkowe zależności zawarte są w~Rozdziale \myrefx{sec:instrukcja-instalacji}. Główne komponenty na które składa się dołączony program to:

\begin{itemize}
\item Plik \texttt{utils.py};
  \item Plik \texttt{config.py};
  \item Plik  \texttt{model.py};
  \item Zbiory danych \textit{low} i \textit{high};
  \item Skrypty do trenowania sieci;
  \item Skrypt do obliczania skuteczności sieci.
\end{itemize}

\subsection*{Plik \texttt{utils.py}}

Plik ten zawiera wiele funkcji pomocniczych realizujących między innymi:
\begin{itemize}
  \item Wczytanie obrazów z dysku;
  \item Interpretacja anotacji danych;
  \item Zmiana rozdzielczości obrazu;
  \item Pobranie modelu wytrenowanego na zbiorze \textit{COCO} \cite{coco}.
\end{itemize}

\subsection*{Plik \texttt{config.py}}

Plik ten zawiera konfigurację i hiperparametry modelu, między innymi:
\begin{itemize}
  \item Współczynnika uczenia (\textit{learning rate};
  \item Ilość wykorzystywanych kart graficznych;
  \item Liczba klas (w przypadku niniejszej pracy są dwie klasy, to znaczy kort badmintona oraz brak kortu).
\end{itemize}

\subsection*{Plik \texttt{model.py}}

Plik ten zawiera klasy i funkcje służące do stworzenia modelu \textit{Mask R-CNN}, umożliwiające wytrenowanie go oraz pozwalające na detekcję obszaru kortu na podanych obrazach wejściowych.

\subsection*{Zbiory danych \textit{low} i \textit{high}}

Dołączone zbiory danych \textit{low} i \textit{high} zostały opisane w~Rozdziale \myrefx{sec:zbiory}.
Do każdego z nich został dołączony plik \texttt{via\_region\_data.json} zawierający anotacje kortów w obrazie, stworzony za pomocą anotatora opisanego w \myrefx{sec:anotator}.

\subsection*{Skrypt do trenowania sieci}

Dołączony skrypt do trenowania sieci \texttt{train.sh} służy do uruchamiania jednego lub więcej treningów, jeden po drugim. Taka seria treningów uruchamiana jest na podstawie zestawu plików opisujących dany trening - czyli precyzujących na jakim zbiorze danych ma zostać przeprowadzony trening, jaki powinien być podział na podzbiory treningowe i walidacyjne wybranego zbioru, oraz czy trenować oryginalną implementację \textit{Mask R-CNN} czy wersję zmodyfikowaną. Skrypt posiada funkcjonalność powiadamiania za pomocą wysyłania poczty elektronicznej - wiadomość wysyłana jest po każdym zakończonym treningu, z informacją czy zakończył się sukcecem lub czy wystąpił błąd.

\subsection*{Skrypt do obliczania skuteczności}

Dołączony skrypt do trenowania sieci \texttt{validate.sh} operuje na wynikach skryptu do trenowania sieci. Dla każdego przeprowadzonego treningu, uruchamia on wytrenowany model i sprawdza jego działanie na wszystkich podzbiorach z osobna, to znaczy na podzbiorach treningowych, walidacyjnych i testowych zbioru \textit{low} i \textit{high}. Na podstawie wyników na zbiorach walidacyjnych i testowych wyliczane są wyniki opisane w~Rozdziale \myrefx{sec:miary}, natomiast wyniki sieci na podzbiorze testowym są zapisywane na dysku jako bazowy obraz z nałożoną przezroczystą maską przewidzianego kortu, na potrzeby ręcznej weryfikacji skuteczności.
