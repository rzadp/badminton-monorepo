% !TeX root = ../../main.tex

W niniejszym rozdziale przedstawione zostaną zbiory danych, szczegóły przeprowadzenia treningu oraz wyniki następujących eksperymentów:

\begin{itemize}
 \item Trening oryginalnej implementacji Mask R-CNN;
 \item Trening zmodyfikowanej implemetacji Mask R-CNN z dokładniejszą maską;
 \item Użycie generatora sztucznych danych.
\end{itemize}

Wyniki przedstawione w niniejszym rozdziale mogą zostać zreprodukowane z pomocą instrukcji~\footnote{\nameref{sec:instrukcja-instalacji}}~\footnote{\nameref{sec:instrukcja-uzytkownika}}.

\section{Hiperparametry treningu}

Treningi przeprowadzone w ramach niniejszego rozdziału korzystają z następującej strategii doboru współczynnika uczenia (\textit{learning rate}):
\begin{enumerate}
  \item Wartość początkowa: 0.001;
  \item W przypadku braku poprawy skuteczności sieci na zbiorze walidacyjnym w 5 kolejnych epokach treningowych, parametr zmniejsza się pięciokrotnie (ale nie mniej niż wartość minimalna 0.0000001);
  \item Uczenie przerywane jest w przypadku braku poprawy skuteczności w 30 kolejnych epokach treningowych.
\end{enumerate}

Próby wskazały na korzystny wpływ zmniejszania parametru w powyższy sposób na ostateczny wynik.
