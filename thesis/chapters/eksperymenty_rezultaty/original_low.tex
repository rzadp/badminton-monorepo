% !TeX root = ../../main.tex
\newpage
\section{Wyniki z~użyciem sieci Mask R-CNN}

\begin{table}[!h]
	\centering
	\caption{Wyniki na zbiorze \textit{low}}
	\vspace{6pt}
	{\footnotesize
		\begin{tabular}{|c|c|c|c|c|}
			\hline \textbackslash & True Positive & False Positive & False Negative & True Negative \\
      \hline Worst & 2 (40\%) & 3 & 4 & 5 \\
      \hline Best & 2 & 3 & 4 & 5 \\
      \hline Median & 2 & 3 & 4 & 5 \\
      \hline Average & 2 & 3 & 4 & 5 \\
      \hline
		\end{tabular}
	}
	\vspace{0pt}
\end{table}

\vspace{1cm}

\begin{table}[!h]
	\centering
	\caption{Złożone wyniki \textit{low}}
	\vspace{6pt}
	{\footnotesize
		\begin{tabular}{|c|c|c|c|c|}
			\hline \textbackslash & Accuracy & Precision & Sensitivity & Specificity \\
      \hline Worst & 2 (40\%) & 3 & 4 & 5 \\
      \hline Best & 2 & 3 & 4 & 5 \\
      \hline Median & 2 & 3 & 4 & 5 \\
      \hline Average & 2 & 3 & 4 & 5 \\
      \hline
		\end{tabular}
	}
	\vspace{0pt}
\end{table}

\missingfigure{
  Tu będzie wykres rezultatów \textbf{niezmodyfikowanej} sieci Mask R-CNN \newline
  Oś X: Liczba epok treningowych \newline
  Oś Y: Poziom błędu na treningowym i na walidacyjnym  \newline
}

\missingfigure{Tu będzie screenshot ilustrujący problemu ``Falbanek''}
