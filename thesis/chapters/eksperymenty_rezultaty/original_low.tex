% !TeX root = ../../main.tex
\newpage
\section{Wyniki z~użyciem sieci Mask R-CNN}

\begin{table}[!h]
	\centering
	\caption{Wyniki na zbiorze \textit{low}}
	\vspace{6pt}
	{\footnotesize
		\begin{tabular}{|c|c|c|c|c|}
			\hline \textbackslash & True Positive & False Positive & False Negative & True Negative \\
      \hline Average & 270830.84 (47.23\%) & 131.63 (0.02\%) & 21976.08 (3.83\%) & 280501.45 (48.92\%) \\
      \hline Minimal & 147267 (25.68\%) & 0 (0.0\%) & 10954 (1.91\%) & 63840 (11.13\%) \\
      \hline Maximal & 488682 (85.22\%) & 668 (0.12\%) & 53684 (9.36\%) & 414944 (72.36\%) \\
      \hline Median & 265394.0 (46.28\%) & 87.0 (0.02\%) & 21166.0 (3.69\%) & 285509.0 (49.79\%) \\
      \hline
		\end{tabular}
	}
	\vspace{0pt}
\end{table}

\vspace{1cm}

\begin{table}[!h]
	\centering
	\caption{Złożone wyniki \textit{low}}
	\vspace{6pt}
	{\footnotesize
		\begin{tabular}{|c|c|c|c|c|}
			\hline \textbackslash & Accuracy & Sensitivity & Specificity & Precision \\
      \hline Average & 0.96 & 0.92 & 1.0 & 1.0 \\
      \hline Minimal & 0.91 & 0.88 & 1.0 & 1.0 \\
      \hline Maximal & 0.98 & 0.96 & 1.0 & 1.0 \\
      \hline Median & 0.96 & 0.93 & 1.0 & 1.0 \\
      \hline
		\end{tabular}
	}
	\vspace{0pt}
\end{table}

\missingfigure{
  Tu będzie wykres rezultatów \textbf{niezmodyfikowanej} sieci Mask R-CNN \newline
  Oś X: Liczba epok treningowych \newline
  Oś Y: Poziom błędu na treningowym i na walidacyjnym  \newline
}

\missingfigure{Tu będzie screenshot ilustrujący problemu ``Falbanek''}
