% !TeX root = ../../main.tex
\chapter{Eksperymenty i rezultaty}

\section{Wyniki z~użyciem sieci R-CNN}

\TODO{Tu będą rezultaty trenowania sieci bez żadnych modyfikacji z~mojej strony, na zbiorze zebranych prawdziwych zdjęć z~Internetu i od p. Nowisza}

\section{Wyniki zmodyfikowanej sieci}

\TODO{Tu będą Rezultaty po moich modyfikacjach}

\TODO{Tu będą ilustracje obrazujące problem falbanek}

\section{Wyniki z~wykorzystaniem generatora}
\label{sec:wyniki_generator}

\TODO{
  Tu będzie porównanie trzech różnych rezultatów treningu:

  - na samych prawdziwych

  - na samych sztucznych

  - na prawdziwych+sztucznych
}

\section{Rezultaty generalizacji zagadnienia}
\label{sec:generalizacja}

Praca skupia się na problemie detekcji kortu do badmintona, ale porusza też zagadnienie generalizacji rozwiązania na inne sporty z~kortem, jak na przykład tenis ziemny.
\\

\TODO{Tu będzie próba zgeneralizowania tytułowego problemu na tenis ziemny}
