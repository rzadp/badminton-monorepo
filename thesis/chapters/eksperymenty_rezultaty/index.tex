% !TeX root = ../../main.tex
\chapter{Eksperymenty i rezultaty}

Wyniki przedstawione w niniejszym rozdziale mogą zostać zreprodukowane z pomocą instrukcji~\footnote{\nameref{sec:instrukcja-instalacji}}~\footnote{\nameref{sec:instrukcja-uzytkownika}}.

\section{Wyniki z~użyciem sieci Mask R-CNN}

\missingfigure{
  Tu będzie wykres rezultatów niezmodyfikowanej sieci Mask R-CNN \newline
  Oś X: Liczba epok treningowych \newline
  Oś Y: Poziom błędu na treningowym i na walidacyjnym  \newline
}

\missingfigure{Tu będzie screenshot ilustrujący problemu ``Falbanek''}

\section{Wyniki zmodyfikowanej sieci}
\label{sec:wyniki_zmodyfikowanej}

\missingfigure{
  Tu będzie wykres rezultatów zmodyfikowanej sieci Mask R-CNN \newline
  Na zbiorze bez sztucznych danych
}

\missingfigure{Tu będzie screenshot ilustrujący rozwiązany problem ``Falbanek''}

\section{Wyniki z~wykorzystaniem generatora}
\label{sec:wyniki_generator}

\missingfigure{
  Tu będzie wykres rezultatów zmodyfikowanej sieci Mask R-CNN \newline
  Na zbiorze wzbogaconym o sztuczne dane
}

\section{Rezultaty generalizacji zagadnienia}
\label{sec:generalizacja}

Praca skupia się na problemie detekcji kortu do badmintona, ale porusza też zagadnienie generalizacji rozwiązania na inne sporty z~kortem, jak na przykład tenis ziemny.
\\

\TODO{Tu będzie opis zgeneralizowania tytułowego problemu na tenis ziemny}
