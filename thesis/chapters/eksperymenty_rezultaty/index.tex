% !TeX root = ../../main.tex
\newcommand*{\resultssummary}[1]{
  Wyniki oznaczają, że sieć rozważana w tym punkcie rozpoznaje poprawnie średnio \textbf{#1\%} pikseli w obrazie.
}
\newcommand{\resultsoriginal}{bazowej implementacji sieci \textit{Mask R-CNN}}
\newcommand{\resultsmodified}{zmodyfikowanej implementacji sieci \textit{Mask R-CNN}}
\newcommand{\withgenerated}{, rozszerzonym o sztucznie wygenerowane obrazy treningowe}
\newcommand*{\resultsintro}[3]{W tym punkcie przedstawiono wyniki #1 trenowanej na zbiorze danych \textit{#2}#3.}
\newcommand*{\resultssummaryalt}[2]{W analizowanym przypadku wynik wskazuje, iż dokładność rozpoznawania obszaru kortu jest na poziomie \textbf{#1\%} pikseli na obrazie przy precyzji \textbf{#2\%}.}

\chapter{Eksperymenty i wyniki}
\label{sec:eksperymenty_wyniki}
\subimport{}{wstep}
\subimport{zbiory/}{index}
\subimport{low/}{index}
\newpage
\subimport{high/}{index}
\newpage
\subimport{}{podsumowanie}
