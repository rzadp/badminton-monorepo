% !TeX root = ../../main.tex
\newpage
\section{Zbiory danych}

Do przeprowadzenia eksperymentów użyto dwóch zbiorów danych, pozyskanych od firmy ``BLUE''. Jeden zbiór, zwany dalej zbiorem \textit{high} pochodzi z kamer ustawianych ponad kortem. Drugi zbiór, zwany dalej zbiorem \textit{low}, pochodzi z kamer sytuowanych na podłodze, tuż przy liniach kortu.

\begin{figure}[!htb]
  \minipage{0.45\textwidth}
    \includegraphics[width=\linewidth]{../badminton/datasets/badminton_high/train/test_court2-00002.png}
    \caption{Przykładowy obraz ze zbioru danych \textit{high}}
  \endminipage\hfill
  \minipage{0.45\textwidth}
    \includegraphics[width=\linewidth]{../badminton/datasets/badminton_low/train/1564909032792410075.jpg}
    \caption{Przykładowy obraz ze zbioru danych \textit{low}}
  \endminipage\hfill
\end{figure}

\vspace{1cm}

\begin{table}[!h]
	\centering
	\caption{Szczegóły zbioru \textit{low}}
	\vspace{6pt}
	{\footnotesize
		\begin{tabular}{|c|c|c|c|}
			\hline \textbackslash & Podzbiór treningowy & Podzbiór walidacyjny & Podzbiór testowy \\
      \hline Liczba obrazów & 156 & 51 & 2 \\
      \hline
		\end{tabular}
	}
	\vspace{0pt}
\end{table}

\begin{table}[!h]
	\centering
	\caption{Wymiary obrazów w zbiorze \textit{low}}
	\vspace{6pt}
	{\footnotesize
		\begin{tabular}{|c|c|}
			\hline Szerokość & Wysokość \\
      \hline 896 & 640 \\
      \hline
		\end{tabular}
	}
	\vspace{0pt}
\end{table}
