% !TeX root = ../../main.tex
\subsection{Podsumowanie wyników na zbiorze danych \textit{high}}
Eksperymenty przeprowadzone w tym rozdziale polegały na sprawdzeniu skuteczności (według miar skuteczności sieci przedstawionych w rozdziale \myrefx{sec:miary}) sieci oryginalnej oraz sieci zmodyfikowanej, trenowanych na zbiorze \textit{high}.
Sprawdzono też, jak dodanie sztucznie wygenerowanych rekordów treningowych wpływa na skuteczność sieci.


Wyniki wskazują na lepszą skuteczność sieci zmodyfikowanej w porównaniu do sieci oryginalnej.
Na zbiorze danych \textit{high}, zastosowanie zmodyfikowanej sieci zwiększa średnią \textit{Accuracy} z \textbf{0.967} do \textbf{0.968}.
Na zbiorze \textit{high} wzbogaconym o sztucznie wygenerowane rekordy treningowe, zmodyfikowana sieć w porównaniu do oryginalnej zwiększyła średnią \textit{Accuracy} z \textbf{0.967} do \textbf{0.968}.
Zwiększenie metryki \textit{Accuracy} o 0.001 oznacza o 573 więcej poprawnie rozpoznanych pikseli na obrazach o rozdzielczości 896x640 pikseli.


Interpretacja wyniki pod względem wpływu dodania sztucznie wygenerowanych rekordów treningowych wskazuje na to, że dodanie takich rekordów nie wpływa na średnią \textit{Accuracy} sieci, zarówno w przypadku analizowania sieci oryginalnej jak i sieci zmodyfikowanej.
