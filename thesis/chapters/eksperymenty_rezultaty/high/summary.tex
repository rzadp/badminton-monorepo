% !TeX root = ../../main.tex
\subsection{Podsumowanie wyników na zbiorze danych \textit{high\_87\_10}}
Eksperymenty przeprowadzone w tym rozdziale polegały na sprawdzeniu skuteczności (według miar skuteczności sieci przedstawionych w~Rozdziale \myrefx{sec:miary}) sieci oryginalnej oraz sieci zmodyfikowanej, trenowanych na zbiorze \textit{high\_87\_10}.
Sprawdzono też, jak dodanie sztucznie wygenerowanych rekordów treningowych wpływa na skuteczność sieci.

Wyniki wskazują na nieznacznie lepszą dokładność sieci zmodyfikowanej w~porównaniu do sieci oryginalnej.
Na zbiorze danych \textit{lhigh\_87\_10}, zastosowanie zmodyfikowanej sieci zwiększa średnią dokładność z~\textbf{98.3\%} do \textbf{98.8\%}, natomiast na zbiorze \textit{high\_87\_10} rozszerzonym o sztucznie wygenerowane rekordy treningowe zmodyfikowana metoda zwiększa średnią dokładność z~\textbf{98.4\%} do \textbf{98.8\%}. 
Zmodyfikowana implementacja osiąga także większą czułość - 3.8 punktów procentowych więcej w~porównaniu do sieci oryginalnej na zbiorze danych \textit{low\_186\_21}, oraz 2.5 punktów procentowych więcej w przypadku zbioru \textit{low\_186\_21} rozszerzonego o sztuczne dane.
Jednak nie oznacza to jednoznacznej przewagi jednej metody nad drugą, ponieważ zmodyfikowana metoda osiąga nieznacznie niższą (w granicach 0.1-0.3 punkta procentowego) średnią swoistość i~precyzję w przypadku obu analizowanych zbiorów.


Dodanie sztucznie wygenerowanych rekordów treningowych wskazuje na fakt, iż dodanie takich rekordów nie wpływa na średnią dokładność sieci, zarówno w przypadku analizowania sieci oryginalnej, jak i sieci zmodyfikowanej. W przypadku oryginalnej implementacji dodanie sztucznych danych treningowych nie wpłynęło na średnią dokładność sieci, która pozostała na poziomie \textbf{98.8\%}. W przypadku zmodyfikowanej implementacji osiągnięto marginalny (\textbf{0.1\%}) wzrost dokładności.


Podsumowując eksperymenty, najlepszy wynik (średnia dokładność \textbf{98.8\%} przy precyzji \textbf{98.1\%}) osiągnięto poprzez zastosowanie zmodyfikowanej implementacji sieci, trenowanej na zbiorze \textit{high\_87\_10} rozszerzonym o sztucznie wygenerowane dane treningowe.
