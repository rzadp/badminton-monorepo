% !TeX root = ../../main.tex
\subsection{Sieć \textit{Mask R-CNN} trenowana na zbiorze \textit{high\_87\_10}}
\label{sec:results_high_original}
\resultsintro{\resultsoriginal}{high\_87\_10}{}
\highresultstables{Tab:high_original_calculated}

\begin{table}[H]
	\centering
	\caption{Wyniki sieci \textit{Mask R-CNN} na zbiorze \textit{high\_87\_10} - przy użyciu metryk z Rozdziału \numberref{sec:miary}}
	\vspace{6pt}
	{\footnotesize
		\begin{tabular}{|c|c|c|c|c|}
      \hline \textbackslash & Dokładność & Czułość & Swoistość & Precyzja \\
      \hline Średnia & 0.967 & 0.829 & 0.988 & 0.890 \\
      \hline Minimum & 0.804 & 0.026 & 0.894 & 0.027 \\
      \hline Maksimum & 0.993 & 0.936 & 1.000 & 1.000 \\
      \hline Mediana & 0.984 & 0.918 & 0.997 & 0.987 \\
      \hline
		\end{tabular}
	}
  \vspace{0pt}
  \label{Tab:high_original_calculated}
\end{table}

\resultssummaryalt{96.7}{89.0}
Warto zauważyć, iż minimalna czułość i swoistość jest o wiele niższa niż w porównaniu do wyników na zbiorze \textit{low\_186\_21}, co wynika z tego, iż korty w obrazach zbioru \textit{high} są o wiele trudniejsze do wykrycia z powodu większej ilości obiektów w obrazach, bardziej różnorodnej perspektywie oraz z powodu rekordów z więcej niż jednym kortem.
