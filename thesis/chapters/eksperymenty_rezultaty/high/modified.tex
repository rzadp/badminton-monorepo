% !TeX root = ../../main.tex
\subsection{Zmodyfikowana sieć trenowana na zbiorze \textit{high\_87\_10}}
\label{sec:results_high_modified}
\resultsintro{\resultsmodified}{high\_87\_10}{}
\highresultstables{Tab:high_modified_calculated}

\begin{table}[H]
	\centering
	\caption{Wyniki zmodyfikowanej sieci \textit{Mask R-CNN} na zbiorze \textit{high\_87\_10} - przy użyciu metryk z~Rozdziału \numberref{sec:miary}}
	\vspace{6pt}
	{\footnotesize
		\begin{tabular}{|c|c|c|c|c|}
      \hline \textbackslash & Dokładność & Czułość & Swoistość & Precyzja \\
      \hline Średnia & 0.988 & 0.958 & 0.994 & 0.971 \\
      \hline Minimum & 0.982 & 0.922 & 0.990 & 0.928 \\
      \hline Maksimum & 0.993 & 0.985 & 0.999 & 0.997 \\
      \hline Mediana & 0.988 & 0.957 & 0.993 & 0.972 \\
      \hline
		\end{tabular}
	}
  \vspace{0pt}
  \label{Tab:high_modified_calculated}
\end{table}

\resultssummaryalt{98.8}{97.1}
W stosunku do wyników z~Rozdziału \numberref{sec:results_high_original} wyniki są marginalnie lepsze jeśli chodzi o dokładność. Średnia czułość okazała się wyższa o \textbf{3.8} punktów procentowych.
