% !TeX root = ../../main.tex
\subsection{Sieć \textit{Mask R-CNN} trenowana na zbiorze \textit{high\_87\_10} ze sztucznymi obrazami}
\label{sec:results_high_original_generated}
\resultsintro{\resultsoriginal}{high\_87\_10}{\withgenerated}
\highresultstables{Tab:high_original_generated_calculated}

\begin{table}[H]
	\centering
	\caption{Wyniki sieci \textit{Mask R-CNN} na zbiorze \textit{high\_87\_10} ze sztucznymi obrazami - przy użyciu metryk z Rozdziału \numberref{sec:miary}}
	\vspace{6pt}
	{\footnotesize
		\begin{tabular}{|c|c|c|c|c|}
      \hline \textbackslash & Dokładność & Czułość & Swoistość & Precyzja \\
      \hline Średnia & 0.984 & 0.921 & 0.998 & 0.990 \\
      \hline Minimum & 0.980 & 0.890 & 0.995 & 0.977 \\
      \hline Maksimum & 0.991 & 0.943 & 1.000 & 1.000 \\
      \hline Mediana & 0.984 & 0.923 & 0.999 & 0.993 \\
      \hline
		\end{tabular}
	}
  \vspace{0pt}
  \label{Tab:high_original_generated_calculated}
\end{table}

\resultssummaryalt{98.4}{99.0}
W porównaniu z wynikami z Rozdziału \numberref{sec:results_high_original}, wyniki analizowanego przypadku różnią się tylko marginalnie. Wskazuje to na to, że rozszerzenie zbioru \textit{high\_87\_10} o sztucznie wygenerowane obrazy nie poprawia wyniku analizowanej metody.
