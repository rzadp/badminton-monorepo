% !TeX root = ../../main.tex
\section{Podsumowanie wyników}

Eksperymenty przeprowadzone w niniejszym rozdziale obejmowały:
\begin{itemize}
 \item zbadanie wpływu podziału zbiorów danych na skuteczność detekcji obszaru kortu;
 \item zbadanie skuteczności zmodyfikowanej implementacji sieci \textit{Mask R-CNN} w~porównaniu do oryginalnej implementacji tej metody;
 \item zbadanie wpływu dodania sztucznie wygenerowanych rekordów treningowych na skuteczność detekcji obszaru kortu;
 \item zbadanie skuteczności połączenia aglorytmicznej metody wykorzystywanej w firmie \blue{} z modelem zmodyfikowanej \textit{Mask R-CNN}.
\end{itemize}

Eksperymenty z podziałem zbiorów danych \textit{high} i \textit{low} (patrz Rozdział \numberref{sec:zbiory}) wykazały, iż najlepszym podziałem jest podział na podzbiór treningowy i walidacyjny w stosunku odpowiednio 90:10. Ze względu na najwyższą dokładność obliczaną na zbiorze testowym, taki podział stosowany był w kolejnych eksperymentach. Zbiory podzielone w taki sposób nazwano odpowiednio \textit{high\_87\_10} i \textit{low\_186\_21}.

Eksperymenty z trenowaniem sieci na zbiorze \textit{low\_186\_21}, zbiorze \textit{high\_87\_10} oraz na wersjach tych zbiorów rozszerzonych o sztucznie wygenerowane rekordy treningowych wykazały, iż dodanie takich rekordów zwiększaja one skuteczność wytrenowanej sieci (zarówno oryginalnej, jak i zmodyfikowanej) na zbiorze \textit{low\_186\_21}. Nie mają one jednak wpływu na wynik sieci na zbiorze \textit{high\_87\_10}. Wynika to z~faktu, iż obrazy ze zbioru \textit{low} są mało zróżnicowane; obrazy wyglądają bardzo podobnie, są w~skali szarości. Ponadto rzadko kort jest zasłaniany przez inne obiekty. Zbiór dużo zyskuje na dodaniu zróżnicowanych, kolorowych obrazów z obiektami zasłaniającymi kort. Natomiast obrazy ze zbioru \textit{high} są zróżnicowane - obrazy pochodzą z różnych źródeł i znacząco różnią się od siebie. Prawdopodobnie dlatego dodanie sztucznych rekordów nie wpływa aż tak na skuteczność sieci trenowanej na takim zbiorze.


Najlepszy wynik (średnia dokładność na poziomie \textbf{98.8\%}) w~przypadku obrazów ze zbioru \textit{high\_87\_10} osiągnięto przy użyciu zmodyfikowanej implementacji \textit{Mask R-CNN} wytrenowanej na zbiorze \textit{high\_87\_10} rozszerzonym o sztuczne rekordy treningowe. Podobnie w~przypadku obrazów ze zbioru \textit{low\_186\_21} najlepszy wynik (średnia dokładność na poziomie \textbf{97.7\%}) osiągnięto przy użyciu zmodyfikowanej implementacji \textit{Mask R-CNN} wytrenowanej na zbiorze \textit{low\_186\_21} rozszerzonym o sztuczne rekordy treningowe. Wyższe wyniki zmodyfikowanej implementacji spowodowane są wiekszą rozdzielczością maski generowanej przez sieć, co pozwala osiągnąć wyższą dokładność detekcji obszaru kortu.

Eksperymenty przeprowadzone przez firmę \blue{} miały na celu zbadanie skuteczności dotychczasowej algorytmicznej metody detekcji obszaru kortu rozszerzonej o zaproponowaną w~niniejszej pracy metodę z użyciem sieci neuronowej. Wyniki wskazują na znaczący (\textbf{10.6} punktów procentowych) wzrost dokładności takiej łączonej metody w~porównaniu do dotychczasowego rozwiązania.
