% !TeX root = ../../main.tex
\section{Eksperymenty dotyczące podziału danych na podzbioru}
\label{sec:podzial_eksperyment}

W tym rozdziale przedstawiono wyniki eksperymentów dotyczące podziału zbiorów na podzbiory treningowe i walidacyjne.
Zbiór treningowy używany jest do dopasowania wag sieci podczas treningu.
Na zbiorze walidacyjnym, rozłączny względem zbioru treningowego, obliczana jest skuteczność sieci.
Dzięki temu że zbiór walidacyjny i zbiór treningowy są rozłączne możliwe jest wychwycenie sytuacji przetrenowania sieci, czyli przypadku gdy wagi sieci zbyt mocno dopasowały się do do detekcji obszaru kortu w rekordach zbioru treningowego, przez co sieć osiąga gorsze wyniki na rekordach niewchodzących w skład zbioru treningowego.
Zbiór testowy wykorzystywany jest do ręcznego sprawdzenia skuteczności sieci, tzn. do sprawdzenia generowanej maski na zapisanym na dysku obrazie.

Eksperymenty miały na celu wybór podziału zbiorów \textit{low} i \textit{high} dającego możliwie jak najlepsze rezultaty. Zbiory opisano w~Rozdziale \numberref{sec:zbiory}, natomiast rozważane podziały zbiorów opisano w Tabeli \numberref{Tab:podzial}.
