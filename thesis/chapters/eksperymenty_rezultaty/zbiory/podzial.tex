% !TeX root = ../../main.tex
\section{Podzbiór treningowy, walidacyjny i testowy}
\label{sec:podzial}
W tym rozdziale omówiono podział zbiorów danych na podzbiory treningowe, walidacyjne i testowe.
Ze względu na stosunkowo małą liczność zbiorów danych, podzbiór testowy został ograniczony do dwóch obrazów.
Pozostałe obrazy, wchodzące w skład podzbiorów treningowych i walidacyjnych zostały rozdzielone po przeprowadzeniu eksperymentów tak aby wybrany podział dawał możliwie jak najlepsze wyniki.
Podział zbiorów danych przedstawiono w Tabeli \mytabref{Tab:podzial}.

\begin{table}[H]
	\centering
	\caption{Liczba obrazów w podziale na poszczególne podzbiory}
	\vspace{6pt}
	{\footnotesize
		\begin{tabular}{|c|c|c|c|}
			\hline \textbackslash & Podzbiór treningowy & Podzbiór walidacyjny & Podzbiór testowy \\
      \hline Zbiór \textit{high} & 92 (94.85\%) & 5 (5.15\%) & 2 (2.06\%) \\
      \hline Zbiór \textit{high} & 87 (89.69\%) & 10 (10.31\%) & 2 (2.06\%) \\
      \hline Zbiór \textit{high} & 73 (75.26\%) & 24 (24.74\%) & 2 (2.06\%) \\
      \hline Zbiór \textit{high} & 49 (50.52\%) & 48 (49.48\%) & 2 (2.06\%) \\
      \hline Zbiór \textit{low} & 197 (95.17\%) & 10 (4.83\%) & 2 (0.97\%) \\
      \hline Zbiór \textit{low} & 186 (89.86\%) & 21 (10.14\%) & 2 (0.97\%) \\
      \hline Zbiór \textit{low} & 155 (74.88\%) & 52 (25.12\%) & 2 (0.97\%) \\
      \hline Zbiór \textit{low} & 104 (50.24\%) & 103 (49.76\%) & 2 (0.97\%) \\
      \hline
    \end{tabular}
    \label{Tab:podzial}
	}
	\vspace{0pt}
\end{table}


W kolejnych podrozdziałach przedstawiono wyniki sieci trenowanej z użyciem podzbiorów podzielonych według poszczególnych podziałów.
Miary skuteczności \textit{Dokładność}, \textit{Czułość}, \textit{Swoistość}, oraz \textit{Precyzja} zostały opisane w rozdziale \myrefx{sec:miary}.
