% !TeX root = ../../main.tex
\subsubsection{Podział zbioru \textit{high} w stosunku 87:10}

W tym podrozdziale przedstawiono wyniki skuteczności sieci wytrenowanej na zbiorze \textit{high}, podzielonym na podzbiór treningowy i podzbiór walidacyjny w stosunku odpowiednio 87 do 10.

\begin{table}[H]
	\centering
	\caption{Wyniki podziału zbioru \textit{high} w stosunku 87:10 na zbiorze walidacyjnym}
	\vspace{6pt}
	{\footnotesize
		\begin{tabular}{|c|c|c|c|c|}
      \hline \textbackslash & Accuracy & Sensitivity & Specificity & Precision \\
      \hline Średnia & 0.970 & 0.854 & 0.989 & 0.909 \\
      \hline Minimum & 0.797 & 0.000 & 0.890 & 0.000 \\
      \hline Maksimum & 0.992 & 0.956 & 1.000 & 1.000 \\
      \hline Mediana & 0.986 & 0.933 & 0.999 & 0.995 \\
      \hline
		\end{tabular}
	}
	\vspace{0pt}
\end{table}

\begin{table}[H]
	\centering
	\caption{Wyniki podziału zbioru \textit{high} w stosunku 87:10 na zbiorze testowym}
	\vspace{6pt}
	{\footnotesize
		\begin{tabular}{|c|c|c|c|c|}
      \hline \textbackslash & Accuracy & Sensitivity & Specificity & Precision \\
      \hline Średnia & 0.980 & 0.948 & 0.994 & 0.994 \\
      \hline Minimum & 0.975 & 0.933 & 0.987 & 0.988 \\
      \hline Maksimum & 0.984 & 0.964 & 1.000 & 1.000 \\
      \hline Mediana & 0.980 & 0.948 & 0.994 & 0.994 \\
      \hline
		\end{tabular}
	}
	\vspace{0pt}
\end{table}

Wyniki według miary \textit{Accuracy} wyniki wynoszą 0.970 na zbiorze walidacyjnym oraz 0.980 na zbiorze testowym.
