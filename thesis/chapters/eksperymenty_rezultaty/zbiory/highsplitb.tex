% !TeX root = ../../main.tex
\subsection{Podział zbioru \textit{high} w stosunku 87:10}
\label{sec:highsplitb}

\splitresults{Tab:highsplitb_val}{Tab:highsplitb_test}{high}{87:10}{high\_87\_10}

\begin{table}[H]
	\centering
	\caption{\splittableval{high\_87\_10}}
	\vspace{6pt}
	{\footnotesize
		\begin{tabular}{|c|c|c|c|c|}
      \hline \textbackslash & Dokładność & Czułość & Swoistość & Precyzja \\
      \hline Średnia & 0.970 & 0.854 & 0.989 & 0.909 \\
      \hline Minimum & 0.797 & 0.000 & 0.890 & 0.000 \\
      \hline Maksimum & 0.992 & 0.956 & 1.000 & 1.000 \\
      \hline Mediana & 0.986 & 0.933 & 0.999 & 0.995 \\
      \hline
    \end{tabular}
    \label{Tab:highsplitb_val}
	}
	\vspace{0pt}
\end{table}

\begin{table}[H]
	\centering
	\caption{\splittabletest{high\_87\_10}}
	\vspace{6pt}
	{\footnotesize
		\begin{tabular}{|c|c|c|c|c|}
      \hline \textbackslash & Dokładność & Czułość & Swoistość & Precyzja \\
      \hline Średnia & 0.980 & 0.948 & 0.994 & 0.994 \\
      \hline Minimum & 0.975 & 0.933 & 0.987 & 0.988 \\
      \hline Maksimum & 0.984 & 0.964 & 1.000 & 1.000 \\
      \hline Mediana & 0.980 & 0.948 & 0.994 & 0.994 \\
      \hline
    \end{tabular}
    \label{Tab:highsplitb_test}
	}
	\vspace{0pt}
\end{table}

\resultssummarysplit{97.0}{98.0}
Minimum miar czułości i precyzji na poziomie \textbf{0\%} spowodowane jest tym, że na jednym z obrazów walidacyjnych obszar kortu nie został wykryty wcale. Powoduje to niższą wartość dokładności, czułości, swoistości i precyzji liczoną na zbiorze walidacyjnym w porównaniu do wyników na zbiorze testowym.
