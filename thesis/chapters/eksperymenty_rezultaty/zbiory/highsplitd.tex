% !TeX root = ../../main.tex
\subsection{Podział zbioru \textit{high} w stosunku 49:48}
\label{sec:highsplitd}

\splitresults{Tab:highsplitd_val}{Tab:highsplitd_test}{high}{49:48}{high\_49\_48}

\begin{table}[H]
	\centering
	\caption{\splittableval{high\_49\_48}}
	\vspace{6pt}
	{\footnotesize
		\begin{tabular}{|c|c|c|c|c|}
      \hline \textbackslash & Dokładność & Czułość & Swoistość & Precyzja \\
      \hline Średnia & 0.950 & 0.795 & 0.978 & 0.841 \\
      \hline Minimum & 0.678 & 0.000 & 0.813 & 0.000 \\
      \hline Maksimum & 0.990 & 0.963 & 1.000 & 1.000 \\
      \hline Mediana & 0.982 & 0.936 & 0.997 & 0.989 \\
      \hline
    \end{tabular}
    \label{Tab:highsplitd_val}
	}
	\vspace{0pt}
\end{table}

\begin{table}[H]
	\centering
	\caption{\splittabletest{high\_49\_48}}
	\vspace{6pt}
	{\footnotesize
		\begin{tabular}{|c|c|c|c|c|}
      \hline \textbackslash & Dokładność & Czułość & Swoistość & Precyzja \\
      \hline Średnia & 0.977 & 0.950 & 0.989 & 0.987 \\
      \hline Minimum & 0.971 & 0.937 & 0.981 & 0.982 \\
      \hline Maksimum & 0.983 & 0.962 & 0.997 & 0.991 \\
      \hline Mediana & 0.977 & 0.950 & 0.989 & 0.987 \\
      \hline
    \end{tabular}
    \label{Tab:highsplitd_test}
	}
	\vspace{0pt}
\end{table}

\resultssummarysplit{95.0}{97.7}
Podobnie jak w podrozdziale \numberref{sec:highsplitb} i podrozdziale \numberref{sec:highsplitc}, minimum miar czułości i~precyzji na poziomie \textbf{0\%} spowodowane jest tym, że na jednym z obrazów walidacyjnych obszar kortu nie został wykryty wcale.
