% !TeX root = ../../main.tex
\subsubsection{Podział zbioru \textit{high} w stosunku 92:5}

\splitresults{Tab:highsplita_val}{Tab:highsplita_test}{high}{92 do 5}

\begin{table}[H]
	\centering
	\caption{Wyniki podziału zbioru \textit{high} w stosunku 92:5 na zbiorze walidacyjnym}
	\vspace{6pt}
	{\footnotesize
		\begin{tabular}{|c|c|c|c|c|}
      \hline \textbackslash & Dokładność & Czułość & Swoistość & Precyzja \\
      \hline Średnia & 0.988 & 0.921 & 1.000 & 0.998 \\
      \hline Minimum & 0.985 & 0.909 & 0.999 & 0.993 \\
      \hline Maksimum & 0.993 & 0.930 & 1.000 & 1.000 \\
      \hline Mediana & 0.987 & 0.922 & 1.000 & 1.000 \\
      \hline
    \end{tabular}
    \label{Tab:highsplita_val}
	}
	\vspace{0pt}
\end{table}

\begin{table}[H]
	\centering
	\caption{Wyniki podziału zbioru \textit{high} w stosunku 92:5 na zbiorze testowym}
	\vspace{6pt}
	{\footnotesize
		\begin{tabular}{|c|c|c|c|c|}
      \hline \textbackslash & Dokładność & Czułość & Swoistość & Precyzja \\
      \hline Średnia & 0.977 & 0.952 & 0.988 & 0.986 \\
      \hline Minimum & 0.969 & 0.944 & 0.978 & 0.979 \\
      \hline Maksimum & 0.985 & 0.961 & 0.998 & 0.992 \\
      \hline Mediana & 0.977 & 0.952 & 0.988 & 0.986 \\
      \hline
    \end{tabular}
    \label{Tab:highsplita_test}
	}
	\vspace{0pt}
\end{table}

Wyniki według miary \textit{Dokładność} wyniki wynoszą 0.988 na zbiorze walidacyjnym oraz 0.977 na zbiorze testowym.
