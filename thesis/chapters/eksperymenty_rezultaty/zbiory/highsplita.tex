% !TeX root = ../../main.tex
\subsection{Podział zbioru \textit{high} w stosunku 92:5}
\label{sec:highsplita}

\splitresults{Tab:highsplita_val}{Tab:highsplita_test}{high}{92:5}{high\_92\_5}

\begin{table}[H]
	\centering
	\caption{\splittableval{high\_92\_5}}
	\vspace{6pt}
	{\footnotesize
		\begin{tabular}{|c|c|c|c|c|}
      \hline \textbackslash & Dokładność & Czułość & Swoistość & Precyzja \\
      \hline Średnia & 0.988 & 0.921 & 1.000 & 0.998 \\
      \hline Minimum & 0.985 & 0.909 & 0.999 & 0.993 \\
      \hline Maksimum & 0.993 & 0.930 & 1.000 & 1.000 \\
      \hline Mediana & 0.987 & 0.922 & 1.000 & 1.000 \\
      \hline
    \end{tabular}
    \label{Tab:highsplita_val}
	}
	\vspace{0pt}
\end{table}

\begin{table}[H]
	\centering
	\caption{\splittabletest{high\_92\_5}}
	\vspace{6pt}
	{\footnotesize
		\begin{tabular}{|c|c|c|c|c|}
      \hline \textbackslash & Dokładność & Czułość & Swoistość & Precyzja \\
      \hline Średnia & 0.977 & 0.952 & 0.988 & 0.986 \\
      \hline Minimum & 0.969 & 0.944 & 0.978 & 0.979 \\
      \hline Maksimum & 0.985 & 0.961 & 0.998 & 0.992 \\
      \hline Mediana & 0.977 & 0.952 & 0.988 & 0.986 \\
      \hline
    \end{tabular}
    \label{Tab:highsplita_test}
	}
	\vspace{0pt}
\end{table}

\resultssummarysplit{98.8}{97.7}
Warto zauważyć, że wyniki na zbiorze testowym charakteryzują się wyższą średnią czułością w~porównaniu do wyników na zbiorze walidacyjnym, ale przy niższej swoistości i~precyzji.
