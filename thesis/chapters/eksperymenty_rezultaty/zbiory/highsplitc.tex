% !TeX root = ../../main.tex
\subsection{Podział zbioru \textit{high} w stosunku 73:24}
\label{sec:highsplitc}

\splitresults{Tab:highsplitc_val}{Tab:highsplitc_test}{high}{73:24}{high\_73\_24}

\begin{table}[H]
	\centering
	\caption{\splittableval{high\_73\_24}}
	\vspace{6pt}
	{\footnotesize
		\begin{tabular}{|c|c|c|c|c|}
      \hline \textbackslash & Dokładność & Czułość & Swoistość & Precyzja \\
      \hline Średnia & 0.967 & 0.831 & 0.987 & 0.887 \\
      \hline Minimum & 0.798 & 0.000 & 0.890 & 0.000 \\
      \hline Maksimum & 0.993 & 0.935 & 1.000 & 1.000 \\
      \hline Mediana & 0.985 & 0.927 & 0.998 & 0.987 \\
      \hline
    \end{tabular}
    \label{Tab:highsplitc_val}
	}
	\vspace{0pt}
\end{table}

\begin{table}[H]
	\centering
	\caption{\splittabletest{high\_73\_24}}
	\vspace{6pt}
	{\footnotesize
		\begin{tabular}{|c|c|c|c|c|}
      \hline \textbackslash & Dokładność & Czułość & Swoistość & Precyzja \\
      \hline Średnia & 0.977 & 0.952 & 0.990 & 0.988 \\
      \hline Minimum & 0.968 & 0.949 & 0.981 & 0.982 \\
      \hline Maksimum & 0.986 & 0.955 & 0.998 & 0.993 \\
      \hline Mediana & 0.977 & 0.952 & 0.990 & 0.988 \\
      \hline
    \end{tabular}
    \label{Tab:highsplitc_test}
	}
	\vspace{0pt}
\end{table}

\resultssummarysplit{96.7}{97.7}
Podobnie jak w podrozdziale \numberref{sec:highsplitb}, minimum miar czułości i precyzji na poziomie \textbf{0\%} spowodowane jest tym, że na jednym z obrazów walidacyjnych obszar kortu nie został wykryty wcale.
