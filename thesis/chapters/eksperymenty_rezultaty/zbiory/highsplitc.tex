% !TeX root = ../../main.tex
\subsubsection{Podział zbioru \textit{high} w stosunku 73:24}

W tym podrozdziale przedstawiono wyniki skuteczności sieci wytrenowanej na zbiorze \textit{high}, podzielonym na podzbiór treningowy i podzbiór walidacyjny w stosunku odpowiednio 73 do 24.

\begin{table}[H]
	\centering
	\caption{Wyniki podziału zbioru \textit{high} w stosunku 73:24 na zbiorze walidacyjnym}
	\vspace{6pt}
	{\footnotesize
		\begin{tabular}{|c|c|c|c|c|}
      \hline \textbackslash & Accuracy & Sensitivity & Specificity & Precision \\
      \hline Średnia & 0.967 & 0.831 & 0.987 & 0.887 \\
      \hline Minimum & 0.798 & 0.000 & 0.890 & 0.000 \\
      \hline Maksimum & 0.993 & 0.935 & 1.000 & 1.000 \\
      \hline Mediana & 0.985 & 0.927 & 0.998 & 0.987 \\
      \hline
		\end{tabular}
	}
	\vspace{0pt}
\end{table}

\begin{table}[H]
	\centering
	\caption{Wyniki podziału zbioru \textit{high} w stosunku 73:24 na zbiorze testowym}
	\vspace{6pt}
	{\footnotesize
		\begin{tabular}{|c|c|c|c|c|}
      \hline \textbackslash & Accuracy & Sensitivity & Specificity & Precision \\
      \hline Średnia & 0.977 & 0.952 & 0.990 & 0.988 \\
      \hline Minimum & 0.968 & 0.949 & 0.981 & 0.982 \\
      \hline Maksimum & 0.986 & 0.955 & 0.998 & 0.993 \\
      \hline Mediana & 0.977 & 0.952 & 0.990 & 0.988 \\
      \hline
		\end{tabular}
	}
	\vspace{0pt}
\end{table}

Wyniki według miary \textit{Accuracy} wyniki wynoszą 0.967 na zbiorze walidacyjnym oraz 0.977 na zbiorze testowym.
