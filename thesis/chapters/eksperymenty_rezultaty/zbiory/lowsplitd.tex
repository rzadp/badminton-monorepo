% !TeX root = ../../main.tex
\subsubsection{Podział zbioru \textit{low} w stosunku 104:103}

W tym podrozdziale przedstawiono wyniki skuteczności sieci wytrenowanej na zbiorze \textit{low}, podzielonym na podzbiór treningowy i podzbiór walidacyjny w stosunku odpowiednio 104 do 103.

\begin{table}[H]
	\centering
	\caption{Wyniki na zbiorze walidacyjnym}
	\vspace{6pt}
	{\footnotesize
		\begin{tabular}{|c|c|c|c|c|}
      \hline \textbackslash & Accuracy & Sensitivity & Specificity & Precision \\
      \hline Średnia & 0.958 & 0.916 & 0.999 & 0.999 \\
      \hline Minimum & 0.889 & 0.848 & 0.970 & 0.962 \\
      \hline Maksimum & 0.983 & 0.943 & 1.000 & 1.000 \\
      \hline Mediana & 0.961 & 0.919 & 1.000 & 1.000 \\
      \hline
		\end{tabular}
	}
	\vspace{0pt}
\end{table}

\begin{table}[H]
	\centering
	\caption{Wyniki na zbiorze testowym}
	\vspace{6pt}
	{\footnotesize
		\begin{tabular}{|c|c|c|c|c|}
      \hline \textbackslash & Accuracy & Sensitivity & Specificity & Precision \\
      \hline Średnia & 0.950 & 0.911 & 0.999 & 0.999 \\
      \hline Minimum & 0.943 & 0.902 & 0.999 & 0.999 \\
      \hline Maksimum & 0.958 & 0.920 & 1.000 & 1.000 \\
      \hline Mediana & 0.950 & 0.911 & 0.999 & 0.999 \\
      \hline
		\end{tabular}
	}
	\vspace{0pt}
\end{table}

Wyniki według miary \textit{Accuracy} wyniki wynoszą 0.958 na zbiorze walidacyjnym oraz 0.950 na zbiorze testowym.
