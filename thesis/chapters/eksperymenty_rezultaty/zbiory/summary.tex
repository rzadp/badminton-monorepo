% !TeX root = ../../main.tex
\subsubsection{Podsumowanie eksperymentów z podziałami zbiorów danych}

Podsumowując rezultaty uzyskane w poprzednich podrozdziałach, wyniki przedstawiono na wykresie.
Większy stosunek podzbioru treningowego do podzbioru walidacyjnego daje spodziewany wyższy wynik mierzony na podzbiorze walidacyjnym, ale niekoniecznie powoduje wyższy wynik na podzbiorze testowym.
Dla obu zbiorów najwyższy wynik na podzbiorze testowym został uzyskany w przypadku podział podzbioru treningowego i walidacyjnego w stosunku odpowiednio 90 do 10, dlatego też zbiory danych zostały podzielone w takim stosunku.

\pgfplotstableread[row sep=\\,col sep=&]{
  split & high-val & high-test & low-val & low-test \\
  ~95:5 & 0.988 & 0.977 & 0.969 & 0.953 \\
  ~90:10 & 0.970 & 0.980 & 0.966 & 0.961 \\
  ~75:25 & 0.967 & 0.977 & 0.962 & 0.943 \\
  ~50:50 & 0.950 & 0.977 & 0.958 & 0.950 \\
}\splitsummarydata

\vspace{0.5cm}

\begin{tikzpicture}
    \begin{axis}[
            % ybar,
            % bar width=.8cm,
            width=0.85\textwidth,
            height=0.85\textwidth,
            legend style={at={(0.5,0.05)},
                anchor=south,legend columns=2},
            symbolic x coords={~95:5,~90:10,~75:25,~50:50},
            xtick=data,
            nodes near coords,
            nodes near coords align={vertical},
            ymin=0.92,ymax=1,
            ylabel={Średnia accuracy},
            y label style={at={(-0.05,0.5)}},
            ticklabel style={/pgf/number format/precision=3},
            /pgf/number format/.cd,fixed,precision=3,
            y tick label style={
              /pgf/number format/.cd,
                  fixed,
                  fixed zerofill,
                  precision=3,
              /tikz/.cd
            },
        ]
        \addplot table[x=split,y=high-val]{\splitsummarydata};
        \addplot table[x=split,y=high-test]{\splitsummarydata};
        \addplot table[x=split,y=low-val]{\splitsummarydata};
        \addplot table[x=split,y=low-test]{\splitsummarydata};
        \legend{high na walidacyjnym, high na testowym, low na walidacyjnym, low na testowym}
    \end{axis}
\end{tikzpicture}

Liczbowy podział zbiorów na podzbiory treningowe, walidacyjne oraz testowe przedstawia tabela \mytabref{Tab:podzial_summary}.

\begin{table}[!h]
	\centering
	\caption{Liczba obrazów w podziale na poszczególne podzbiory}
	\vspace{6pt}
	{\footnotesize
		\begin{tabular}{|c|c|c|c|}
			\hline \textbackslash & Podzbiór treningowy & Podzbiór walidacyjny & Podzbiór testowy \\
      \hline Zbiór \textit{high} & 87 & 10 & 2 \\
      \hline Zbiór \textit{low} & 186 & 21 & 2 \\
      \hline
    \end{tabular}
    \label{Tab:podzial_summary}
	}
	\vspace{0pt}
\end{table}
