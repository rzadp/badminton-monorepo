% !TeX root = ../../main.tex
\newpage
\subsubsection{summary}

\pgfplotstableread[row sep=\\,col sep=&]{
  split & high-val & high-test & low-val & low-test \\
  ~95:5 & 0.988 & 0.977 & 0.969 & 0.953 \\
  ~90:10 & 0.970 & 0.980 & 0.966 & 0.961 \\
  ~75:25 & 0.967 & 0.977 & 0.962 & 0.943 \\
  ~50:50 & 0.950 & 0.977 & 0.958 & 0.950 \\
}\splitsummarydata

% pgfplotsset{ticklabel style={/pgf/number format/precision=4}, tick scale binop={\times}}

\begin{tikzpicture}
    \begin{axis}[
            % ybar,
            % bar width=.8cm,
            width=0.85\textwidth,
            height=1.0\textwidth,
            legend style={at={(0.5,0.05)},
                anchor=south,legend columns=2},
            symbolic x coords={~95:5,~90:10,~75:25,~50:50},
            xtick=data,
            nodes near coords,
            nodes near coords align={vertical},
            ymin=0.92,ymax=1,
            ylabel={Średnia accuracy},
            y label style={at={(-0.05,0.5)}},
            ticklabel style={/pgf/number format/precision=3},
            /pgf/number format/.cd,fixed,precision=3,
            y tick label style={
              /pgf/number format/.cd,
                  fixed,
                  fixed zerofill,
                  precision=3,
              /tikz/.cd
            },
        ]
        \addplot table[x=split,y=high-val]{\splitsummarydata};
        \addplot table[x=split,y=high-test]{\splitsummarydata};
        \addplot table[x=split,y=low-val]{\splitsummarydata};
        \addplot table[x=split,y=low-test]{\splitsummarydata};
        \legend{high na walidacyjnym, high na testowym, low na walidacyjnym, low na testowym}
    \end{axis}
\end{tikzpicture}
