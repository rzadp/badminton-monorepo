% !TeX root = ../../main.tex
\subsection{Podsumowanie}

Podsumowanie wyników uzyskanych w podrozdziałach \numberref{sec:highsplita}-\numberref{sec:lowsplitd} zestawiono na wykresie na Rysunku \numberref{fig:podzialchart}.
Wyższy stosunek liczności podzbioru treningowego do liczności podzbioru walidacyjnego daje spodziewany wyższy wynik mierzony na podzbiorze walidacyjnym, ale niekoniecznie powoduje wyższy wynik na podzbiorze testowym.
Dla zbiorów danych \textit{low} i \textit{high} najwyższy wynik na podzbiorze testowym został uzyskany w~przypadku podziału podzbioru treningowego i walidacyjnego w stosunku odpowiednio 90:10, dlatego też wykorzystano ten podział zbioru danych do dalszych eksperymentów.

\pgfplotstableread[row sep=\\,col sep=&]{
  split & high-val & high-test & low-val & low-test \\
  ~95:5 & 0.988 & 0.977 & 0.969 & 0.953 \\
  ~90:10 & 0.970 & 0.980 & 0.966 & 0.961 \\
  ~75:25 & 0.967 & 0.977 & 0.962 & 0.943 \\
  ~50:50 & 0.950 & 0.977 & 0.958 & 0.950 \\
}\splitsummarydata

\vspace{0.5cm}

\begin{figure}[!htb]
\centering
\begin{tikzpicture}
    \begin{axis}[
            % ybar,
            % bar width=.8cm,
            width=0.85\textwidth,
            height=0.85\textwidth,
            legend style={at={(0.5,0.05)},
                anchor=south,legend columns=2},
            symbolic x coords={~95:5,~90:10,~75:25,~50:50},
            xtick=data,
            nodes near coords,
            nodes near coords align={vertical},
            ymin=0.92,ymax=1,
            ylabel={Średnia dokładność detekcji obszaru kortu},
            xlabel={Stosunek liczności podzbioru treningowego do liczności podzbioru walidacyjnego},
            y label style={at={(-0.05,0.5)}},
            ticklabel style={/pgf/number format/precision=3},
            /pgf/number format/.cd,fixed,precision=3,
            y tick label style={
              /pgf/number format/.cd,
                  fixed,
                  fixed zerofill,
                  precision=3,
              /tikz/.cd
            },
        ]
        \addplot table[x=split,y=high-val]{\splitsummarydata};
        \addplot table[x=split,y=high-test]{\splitsummarydata};
        \addplot table[x=split,y=low-val]{\splitsummarydata};
        \addplot table[x=split,y=low-test]{\splitsummarydata};
        \legend{high na walidacyjnym, high na testowym, low na walidacyjnym, low na testowym}
    \end{axis}
    \label{}
\end{tikzpicture}
\caption{Wyniki eksperymentów dotyczących podziałów zbiorów danych}
\label{fig:podzialchart}
\end{figure}

Ostateczny liczbowy podział zbiorów na podzbiory treningowe, walidacyjne oraz testowe, na których przeprowadzono dalsze eksperymenty przedstawiono w~Tabeli \mytabref{Tab:podzial_summary}.

\begin{table}[!h]
	\centering
	\caption{Podział zbiorów danych na potrzeby eskperymentów~\numberref{sec:experymenty_low}-\numberref{sec:experymenty_high}}
	\vspace{6pt}
	{\footnotesize
		\begin{tabular}{|c|c|c|c|}
			\hline \textbackslash & Podzbiór treningowy & Podzbiór walidacyjny & Podzbiór testowy \\
      \hline Zbiór \textit{high\_87\_10} & 87 & 10 & 2 \\
      \hline Zbiór \textit{low\_186\_21} & 186 & 21 & 2 \\
      \hline
    \end{tabular}
    \label{Tab:podzial_summary}
	}
	\vspace{0pt}
\end{table}
