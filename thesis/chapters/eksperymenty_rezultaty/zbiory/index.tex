% !TeX root = ../../main.tex

\newcommand*{\splitresults}[5]{W~Tabeli \mytabref{#1} oraz Tabeli \mytabref{#2} przedstawiono wyniki skuteczności detekcji obszaru kortu przy użyciu sieci wytrenowanej na zbiorze \textit{#5}, tzn. na zbiorze \textit{#3} podzielonym na podzbiór treningowy i podzbiór walidacyjny w stosunku odpowiednio #4.}

\newcommand*{\splittableval}[1]{Skuteczność detekcji obszaru kortu na zbiorze walidacyjnym przy użyciu sieci wytrenowanej na zbiorze \textit{#1}}
\newcommand*{\splittabletest}[1]{Skuteczność detekcji obszaru kortu na zbiorze testowym przy użyciu sieci wytrenowanej na zbiorze \textit{#1}}

\newcommand*{\resultssummarysplit}[2]{W analizowanym przypadku otrzymane wyniki wskazują, iż dokładność rozpoznawania obszaru kortu jest na poziomie \textbf{#1\%} poprawnie zdetektowanych pikseli na zbiorze walidacyjnym oraz \textbf{#2\%} na zbiorze testowym.}

\subimport{}{podzial}
\subimport{}{highsplita}
\subimport{}{highsplitb}
\subimport{}{highsplitc}
\subimport{}{highsplitd}
\subimport{}{lowsplita}
\subimport{}{lowsplitb}
\subimport{}{lowsplitc}
\subimport{}{lowsplitd}
\vfill
\subimport{}{summary}
