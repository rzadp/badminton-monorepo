% !TeX root = ../../main.tex
\subsection{Podsumowanie wyników na zbiorze danych \textit{low\_186\_21}}
Eksperymenty przeprowadzone w tym rozdziale polegały na sprawdzeniu skuteczności (według miar skuteczności sieci przedstawionych w~Rozdziale \myrefx{sec:miary}) sieci oryginalnej oraz sieci zmodyfikowanej, trenowanych na zbiorze \textit{low\_186\_21}.
Sprawdzono też, jak dodanie sztucznie wygenerowanych rekordów treningowych wpływa na skuteczność sieci.


Wyniki wskazują na lepszą skuteczność sieci zmodyfikowanej w porównaniu do sieci oryginalnej.
Na zbiorze danych \textit{low\_186\_21}, zastosowanie zmodyfikowanej sieci zwiększa średnią \textit{Dokładność} z \textbf{0.953} do \textbf{0.955}.
Zwiększenie metryki \textit{Dokładność} o 0.002 oznacza o 1147 więcej poprawnie rozpoznanych pikseli w obrazach o rozdzielczości 896x640 pikseli. Na zbiorze \textit{low\_186\_21} wzbogaconym o sztucznie wygenerowane rekordy treningowe, zmodyfikowana sieć w porównaniu do oryginalnej zwiększyła średnią \textit{Dokładność} z \textbf{0.962} do \textbf{0.977}.


Interpretacja wyniki pod względem wpływu dodania sztucznie wygenerowanych rekordów treningowych wskazuje na to, iż dodanie takich rekordów zwiększa średnią skuteczność sieci. W przypadku oryginalnej sieci, dodanie sztucznych rekordów zwiększyło średnią \textit{Dokładność} z \textbf{0.953} do \textbf{0.962}. Rozważając zmodyfikowaną sieć, dodanie sztucznych rekordów zwiększyło średnią \textit{Dokładność} z \textbf{0.955} do \textbf{0.977}.
