% !TeX root = ../../main.tex
\subsection{Podsumowanie wyników na zbiorze danych \textit{low\_186\_21}}
Eksperymenty przeprowadzone w tym rozdziale polegały na sprawdzeniu skuteczności (według miar skuteczności przedstawionych w~Rozdziale \myrefx{sec:miary}) oryginalnej sieci \textit{Mask R-CNN} oraz implementacji zmodyfikowanej, trenowanych na zbiorze \textit{low\_186\_21}.
Sprawdzono też, jak dodanie sztucznie wygenerowanych rekordów treningowych wpływa na skuteczność sieci.


Wyniki wskazują na poprawę dokładności rozpoznawania obszaru kortu przy użyciu sieci zmodyfikowanej w~porównaniu do sieci oryginalnej.
Na zbiorze danych \textit{low\_186\_21}, zastosowanie zmodyfikowanej sieci zwiększa średnią dokładność z \textbf{95.3\%} do \textbf{95.5\%} przy wzroście czułości o \textbf{0.7\%}, natomiast na zbiorze \textit{low\_186\_21} rozszerzonym o sztucznie wygenerowane rekordy treningowe zmodyfikowana metoda zwiększa średnią dokładność z \textbf{96.2\%} do \textbf{97.7\%}. Zwiększenie dokładności o 0.2 punkty procentowe oznacza o 1147 więcej poprawnie rozpoznanych pikseli w obrazach o rozdzielczości 896x640 pikseli.
Podobnie reaguje czułość, to~znaczy zmodyfikowana implementacja osiąga nieznaczny wzrost (0.7 punkta procentowego) w~porównaniu do sieci oryginalnej na zbiorze danych \textit{low\_186\_21}, oraz trochę większy wzrost (3.5 punktów procentowych) w przypadku zbioru \textit{low\_186\_21} rozszerzonego o sztuczne dane. Jednak nie oznacza to jednoznacznej przewagi jednej metody nad drugą, ponieważ zmodyfikowana metoda osiąga nieznacznie niższą (w granicach 0.1-0.3 punkta procentowego) średnią swoistość i~precyzję na zbiorze \textit{low\_186\_21}.


Dodanie sztucznie wygenerowanych rekordów treningowych wskazuje na fakt, iż dodanie takich rekordów zwiększa średnią dokładność sieci. W przypadku oryginalnej sieci, dodanie sztucznych rekordów zwiększyło średnią dokładność z~\textbf{95.3\%} do \textbf{96.2\%}. Potwierdzają to także miary czułości, swoistości i~precyzji. Rozważając natomiast zmodyfikowaną sieć, dodanie sztucznych rekordów zwiększyło średnią dokładność z~\textbf{95.5\%} do \textbf{97.7\%}. W tym przypadku również lepsze wyniki potwierdzone są wzrostem czułości, swoistości i~precyzji. Warto zauważyć, że zastosowanie sztucznych danych treningowych pozwoliło osiągnąć swoistość oraz precyzję na poziomie \textbf{99.9\%}.


Podsumowując ekspetymenty, najlepszy wynik (średnia dokładność \textbf{97.7\%} przy precyzji \textbf{99.9\%}) osiągnięto poprzez zastosowanie zmodyfikowanej implementacji sieci, trenowanej na zbiorze \textit{low\_186\_21} rozszerzonym o sztucznie wygenerowane dane treningowe.
