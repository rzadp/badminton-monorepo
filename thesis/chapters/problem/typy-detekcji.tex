% !TeX root = ../../main.tex
\section{Typy detekcji obiektów na obrazie}
\label{sec:typy_detekcji}

Wykrycie obiektu na obrazie można rozumieć jako cztery podobne problemy.
W kolejności od najbardziej podstawowego do najbardziej szczegółowego zadania są to:

\begin{enumerate}
	\item Klasyfikacja - polega na stwierdzeniu przynależności wykrytego na obrazie obiektu do jednej z wcześniej zdefiniowanych grup;
  \item Semantyczna segmentacja - polega na wskazaniu, które piksele obrazu należą do wykrytego obiektu.
        W przypadku więcej niż jednego obiektu, nie ma rozróżnienia, które piksele dotyczą którego obiektu;
	\item Detekcja obiektów - polega na wskazaniu, gdzie na obrazie znajdują się wykryte obiekty, z~rozróżnieniem poszczególnych obiektów jednak bez wskazania konkretnych pikseli;
	\item Segmentacja instancji - polega na wskazaniu konkretnych pikseli wykrytych obiektów z~rozróżnieniem poszczególnych obiektów.
\end{enumerate}

Niniejsza praca zajmuje się tym ostatnim zadaniem, to znaczy segmentacją instancji kortów do badmintona.
