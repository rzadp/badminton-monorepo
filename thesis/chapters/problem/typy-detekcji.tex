% !TeX root = ../../main.tex
\section{Detekcja obiektów w obrazie}
\label{sec:typy_detekcji}

Detekcja obiektów w obrazie można rozumieć w pięciu aspektach, różniących się między sobą szczegółowością opisu analizowanego obiektu. Te aspekty to klasyfikacja obrazów, detekcja obiektów, semantyczna segmentacja, segmentacja instancji \cite{survey-of-object-classification} oraz segmentacja panoptyczna \cite{panoptic-segmentation}.

\subsection*{Klasyfikacja obrazów}
Klasyfikacja obrazów polega na stwierdzeniu przynależności obiektu (lub obiektów) na podanym obrazie wejściowym do jednej z wcześniej zdefiniowanych grup. Wynikiem jest kategoria (lub lista kategorii) zindentyfikowanych w obrazie obiektów, bez dodatkowych informacji na temat kształtu, rozmiaru czy położenia obiektów. Jedną z metod rozwiązywania tego problemu jest wykorzystanie klasyfikatora opartego na maszynach wektorów nośnych (\textit{support vector machines, SVM}). Rozwiązanie to wymaga ekstrakcji cech (\textit{feature extraction}), na których to dokonywana jest klasyfikacja przy użyciu \textit{SVM}. Metoda ta ma wysoką skuteczność, dobre podłoże teoretyczne i jest szeroko stosowane w wielu dziedzinach nauki \cite{analysis-image-classification}.

Innym przykładem metody rozwiązywania tego problem jest użycie algorytmu drzew decyzyjnych \textit{(decision tree}), skutecznie zaaplikowanego w ramach problemu klasyfikacji zdjęć satelitarnych \cite{decision-image-classifier}. Zaletą tej metody jest to, że wynikowy model jest stosunkowo łatwy do zinterpretowania.

Kolejną alternatywą powyższych metod jest użycie algorytmu \textit{k} najbliższych sąsiadów (\textit{k-nearest neighbors algorithm, k-NN}). Algorytm ten stosowany jest do klasyfikacji obrazów w różnych dziedzinach nauki, między innymi w biologii \cite{analysis-image-classification}.

\subsection*{Detekcja obiektów}
Detekcja obiektów rozszerza aspekt klasyfikacji obrazów. Tak jak w przypadku klasyfikacji obrazów polega na rozpoznaniu kategorii obiektów w obrazie, ale dodatkowym elementem wyniku są obwiednie wskazujące pozycje zdetektowanych obiektów. W aspekcie detekcji obiektów najcześciej wykorzystuje się modele oparte na sieciach neuronowych. Modele oparte na sieciach neuronowych służące do detekcji obiektów które można podzielić na dwie kategorie \cite{survey-deep-learning-object-dection}. Jedna to modele jednokrokowe (\textit{one-stage}), takie jak na przykład \textit{YOLO} \cite{yolo}. Druga kategoria to modele dwukrokowe (\textit{two-stage}), takie jak na przykład \textit{Fast R-CNN} \cite{fast-rcnn}. W modelach dwukrokowych, pierwszym krokiem jest przygotowanie obwiedni potencjalnych obiektów w obrazie, które stanowią bazę do dalszych obliczeń. Tak przygotowane obwiednie są klasyfikowane i poprawiane (lub odrzucane jako niepoprawne). W modelach jednokrokowych nie korzysta się z tak przygotowanych potencjalnych obwiedni obiektów.

\subsection*{Semantyczna segmentacja}
Semantyczna segmentacja polega na klasyfikacji poszczególnych pikseli na obrazie. Każdemu pikselowi w obrazie przypisywana jest jedna ze zdefiniowanych kategorii. Przy takim podejściu każdy piksel w obrazie ma przypisaną jakąś kategorię. Jeżeli w obrazie zdetektowany zostaje więcej niż jeden obiekt należący do jednej kategorii, piksele tych obiektów zostają zaklasyfikowane do tej samej kategorii jednak bez rozróżnienia które piksele należą do którego obiektu.
Zagadnienie może być rozszerzone o możliwość klasyfikacji pikseli do więcej niż jednej kategorii (\textit{multiple class affiliation}) \cite{survey-semantic-segmentation}, na przykład w przypadku gdy mały obiekt częsciowo przysłania większy obiekt w obrazie.

\subsection*{Segmentacja instancji}
Segmentacja instancji może być rozumiana jako rozszerzenie detekcji obiektów, gdzie zamiast obwiedni wynikiem są wskazane piksele obiektu. Różni się ona od semantycznej segmentacji tym, że w przypadku więcej niż jednego obiektu tej samej kategorii występuje rozróżnienie, które piksele należą do którego obiektu. W przeciwieństwie do semantycznej segmentacji, w aspekcie segmentacji instancji niektóre piksele mogą pozostać niezaklasyfikowane do żadnej kategorii.

\subsection*{Segmentacja panoptyczna}
Zagadnienie segmentacja panoptycznej \cite{panoptic-segmentation} może być rozumiane jako połączenie semantycznej segmentacji i segmentacji isntancji. Piksele w obrazie są kategoryzowane z rozróżnieniem na różne obiekty tej samej kategorii, i dodatkowo każdy piksel obrazu musi zostać zakwalifikowany do jednej z kategorii.

\vspace{0.5cm}

W niniejszej pracy skoncentrowano się na zagadnieniu segmentacji instancji kortów do badmintona, ze względu na fakt, iż system automatycznego sędziowania meczów badmintona tworzony przez firmę \blue{} wykorzystywany będzie w placówkach w których kamery mogą obejmować więcej niż jeden kort, dlatego też rozróżnienie konkretnych pikseli jako konkretne korty jest wymagane. Natomiast samo wskazanie kortów, bez wskazywania konkretnych pikseli nie jest wystarczające do rozstrzygania wątpliwości na temat tego czy lotka wypadła czy nie wypadła poza kort. Segmentacja instancji kortu do badmintona jest zwana w dalszej częsci pracy detekcją obszaru kortu.
