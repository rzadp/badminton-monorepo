% !TeX root = ../../main.tex
\subsection{Algorytmiczna detekcja obszaru kortu}
\label{sec:aglorytmiczna_detekcja}

Stosowany przez firmę \blue{} algorytmiczny sposób detekcji obszaru kortu w obrazie został opisany w następujących podpunktach.

\begin{enumerate}
  \item zastosowanie na obrazie filtra rozmywającego (\textit{blur}), aby kolejny filtr miał mniej fałszywie dodatnich wykryć (\textit{false positives});
  \item zastosowanie na obrazie filtra \textit{Canny} do wykrycia krawędzi;
  \item zastosowanie transformacji \textit{Hough} do wykrycia segmentów linii prostych wśród krawędzi;
  \item pogrupowanie wykrytych segmentów linii na grupy podobnych (takich, których kąt nachylenia jest zbliżony i odległości pomiędzy segmentami są niewielkie);
  \item zastąpienie grup segmentów linii pojedynczymi liniami;
  \item znalezienie i odfiltrowanie (pozostawienie tylko takich) par linii, które:
        \begin{enumerate}
          \item są równoległe lub prawie równoległe;
          \item są blisko siebie w obszarze obrazu;
          \item kolor pomiędzy liniami jest jaśniejszy niż na zewnątrz;
          \item segmenty linii nie przecinają się.
        \end{enumerate}
  \item dla znalezionych par wyliczenie  ``linii środkowych'' leżących dokładnie pomiędzy liniamii tworzącymi pary;
  \item wydłużenie segmentów ``linii środkowych'', tak by leżały na krawędziach obrazu;
  \item usunięcie zdegenerowanych przypadków par linii:
        \begin{enumerate}
          \item usunięcie par linii z przecinającymi się krawędziami;
          \item usunięcie par linii ze złymi kątami pomiędzy krawędziami;
          \item usunięcie par linii ze złymi kolorami pomiędzy krawędziami;
          \item usunięcie par linii, które częściowo się pokrywają (zostawiamy jedną, lepszą parę);
          \item usunięcie linii współdzielących jedną z krawędzi (zostawiamy jedną, lepszą parę).
        \end{enumerate}
  \item znalezienie przecięć par linii z innymi parami linii;
  \item skrócenie par linii do skrajnych przecięć;
  \item podział przecinających się linii na 2 grupy: poziome i pionowe;
  \item weryfikacja liczby pionowych linii (kamera zawsze widzi tylko 1 lub 2 pionowe linie - każdy inny wynik jest błędny).
\end{enumerate}

Dokładność detekcji obszaru kortu przy wykorzystaniu tego podejścia otrzymano na poziomie średnio \textbf{96.97\%}, gdzie średnia dokładność $D$ liczona jest w następujący sposób:
\vspace{-3pt} 
\[
\vspace{-3pt}
D = \ddfrac{\sum_{n = 1}^{|N|} \frac{A(i)}{W_{i} * H_{i}}}{|N|} * 100\%
\vspace{-3pt}
\]
gdzie $N$ jest zbiorem walidacyjnym na którym przeprowadzane są obliczenia, $W_{i}$ oraz $H_{i}$ są odpowiednio szerokością i wysokością w pikselach $i$-tego obrazu, a $A(i)$ jest liczbą poprawnie zidentyfikowanych jako obszar kortu lub jako otoczenie pikseli na $i$-tym obrazie.

Jednak rozwiązanie okazuje się niewystarczające na potrzeby firmy \blue{}, ze względu na fakt, iż w niektórych przypadkach kort nie zostaje wykryty wcale. Dzieje się tak często gdy w pobliżu kortu znajduje się podłużne obiekty, który zostają fałszywie rozpoznane jako linia kortu.

% \begin{figure}[h]
%   \centering
%   \caption{Schemat algorytmicznej detekcji obszaru kortu}
%   \includegraphics[width=0.6\textwidth]{cpp.png}
%   \label{fig:algcpp}
% \end{figure}
