% !TeX root = ../../main.tex
\subsection{Algorytmiczna detekcja obszaru kortu}
\label{sec:aglorytmiczna_detekcja}

Stosowany przez firmę \blue{} algorytmiczny sposób detekcji obszaru kortu na obrazie przedstawiony został za pomocą schematu blokowego na Rysunku \myfigref{fig:algcpp}. Wysokopoziomowy opis algorytmu został przedstawiony w następujących podpunktach.

\begin{itemize}
  \item Rozmazanie obrazu wejściowego;
  \item Detekcja krawędzi;
  \item Detekcja odcinków na podstawie krawędzi;
  \item Wydłużenie odcinków;
  \item Usunięcie zdegradowanych odcinków;
  \item Wybranie najlepszej perspektywy;
  \item Wskazanie pionowych i poziomych linii kortu;
  \item Wskazanie obszaru kortu na obrazie.
\end{itemize}

Dokładność detekcji obszaru kortu przy wykorzystaniu tego podejścia otrzymano na poziomie średnio \textbf{96.97\%}, gdzie średnia dokładność $D$ liczona jest w następujący sposób:
\vspace{-3pt} 
\[
\vspace{-3pt}
D = \ddfrac{\sum_{n = 1}^{|N|} \frac{A(i)}{W_{i} * H_{i}}}{|N|} * 100\%
\vspace{-3pt}
\]
gdzie $N$ jest zbiorem walidacyjnym na którym przeprowadzane są obliczenia, $W_{i}$ oraz $H_{i}$ są odpowiednio szerokością i wysokością w pikselach $i$-tego obrazu, a $A(i)$ jest liczbą poprawnie zidentyfikowanych jako obszar kortu lub jako otoczenie pikseli na $i$-tym obrazie.

Jednak rozwiązanie okazuje się niewystarczające na potrzeby firmy \blue{}, ze względu na fakt, iż w niektórych przypadkach kort nie zostaje wykryty wcale. Dzieje się tak często gdy w pobliżu kortu znajduje się podłużne obiekty, który zostają fałszywie rozpoznane jako linia kortu.

\begin{figure}[h]
  \centering
  \caption{Schemat algorytmicznej detekcji obszaru kortu}
  \includegraphics[width=0.7\textwidth]{cpp.png}
  \label{fig:algcpp}
\end{figure}
