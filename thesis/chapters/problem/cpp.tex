% !TeX root = ../../main.tex
\subsection{Algorytmiczna detekcja obszaru kortu}

Skuteczność podejścia, według miary skuteczności \textit{Accuracy} zaproponowanej w rozdziale \myrefx{sec:miary}, wynosi średnio \textbf{0.9697}.

Wysokopoziomowy opis algorytmu został zebrany w następujących podpunktach. Schemat blokowy algorytmu przedstawiony jest na Rysunku \myfigref{fig:algcpp}.

\begin{itemize}
  \item Rozmazanie obrazu wejściowego
  \item Wykrycie krawędzi
  \item Wykrycie odcinków na podstawie krawędzi
  \item Wydłużenie odcinków
  \item Usunięcie zdegradowanych odcinków
  \item Wybranie najlepszej perspektywy
  \item Wskazanie pionowych i poziomych linii kortu
  \item Wskazanie kortu na obrazie
\end{itemize}

Rozwiązanie okazuje się niewystarczające na potrzeby firmy \blue{}, ze względu na to iż w niektórych przypadkach kort nie zostaje wykryty wcale.

\begin{figure}[h]
  \centering
  \caption{Schemat algorytmicznej detekcji obszaru kortu}
  \includegraphics[width=0.5\textwidth]{cpp.png}
  \label{fig:algcpp}
\end{figure}
