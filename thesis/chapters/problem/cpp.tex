% !TeX root = ../../main.tex
\subsection{Algorytmiczna detekcja obszaru kortu}
\label{sec:aglorytmiczna_detekcja}

Stosowany przez firmę \blue{} algorytmiczny sposób detekcji obszaru kortu na obrazie przedstawiony został za pomocą schematu blokowego na rysunku \myfigref{fig:algcpp}. Wysokopoziomowy opis algorytmu został przedstawiony w następujących podpunktach.

\begin{itemize}
  \item Rozmazanie obrazu wejściowego;
  \item Detekcja krawędzi;
  \item Detekcja odcinków na podstawie krawędzi;
  \item Wydłużenie odcinków;
  \item Usunięcie zdegradowanych odcinków;
  \item Wybranie najlepszej perspektywy;
  \item Wskazanie pionowych i poziomych linii kortu;
  \item Wskazanie obszaru kortu na obrazie.
\end{itemize}

Skuteczność podejścia, według miary skuteczności \textit{Dokładność} zaproponowanej w rozdziale \myrefx{sec:miary}, wynosi średnio \textbf{0.9697}.
Rozwiązanie okazuje się niewystarczające na potrzeby firmy \blue{}, ze względu na to, iż w niektórych przypadkach kort nie zostaje wykryty wcale.

\begin{figure}[h]
  \centering
  \caption{Schemat algorytmicznej detekcji obszaru kortu}
  \includegraphics[width=0.5\textwidth]{cpp.png}
  \label{fig:algcpp}
\end{figure}
