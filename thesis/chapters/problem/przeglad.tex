% !TeX root = ../../main.tex
\subsection{Przegląd sieci neuronowych w kontekście segmentacji instancji}

Do rozwiązania problemu detekcji rozważono kilka modeli open-source:

\begin{itemize}
	\item \textit{Keras RetinaNet} \cite{keras-retinanet}
	\item \textit{YOLO} \cite{yolo}
	\item \textit{DeepMask/SharpMask} \cite{deep-sharp-mask}
	\item Różne implementacje modelu Mask R-CNN \cite{general-mask-rcnn}
		\begin{itemize}
			\item \textit{Mask\_RCNN} \cite{matterport-mask-rcnn}
			\item \textit{Instance segmentation with OpenCV} \cite{mask-rcnn-opencv}
			\item \textit{FastMaskRCNN} \cite{fast-mask-rcnn}
		\end{itemize}
\end{itemize}

Modele \textit{Keras RetinaNet} oraz \textit{YOLO} zostały odrzucone ze względu na fakt, iż pozwalają tylko na detekcję obiektów, nie posiadają funkcjonalności segmentacji instacji. \\

Model \textit{DeepMask/SharpMask} został odrzucony, ponieważ repozytorium jest zarchiwizowane i nie rozwijane od 2016 roku.

Z~pozostałych trzech różnych implementacji bazujących na pracy Mask R-CNN, wybór padł na \textit{Mask\_RCNN} z~jednego, ważnego powodu - autor udostępnia wagi modelu wytrenowanego na publiczym zbiorze zaanotowanych obrazów COCO\footnote{Common Objects in Context, http://cocodataset.org}.
Przydatność tych wag będzie opisana w dalszej części pracy.
