% !TeX root = ../../main.tex
\subsection{Sieci neuronowe w kontekście detekcji obiektów}

w~Rozdziale \numberref{sec:typy_detekcji} przedstawiono różne aspekty problemu detekcji obiektów, to znaczy klasyfikację obrazów, detekcję obiektów, segmentację semantyczną, segmentację instancji oraz segmentację panoptyczną. W ramach każdego z tych aspektów wykorzystywane są modele oparte na sieciach neuronowych. Wśród takich modeli, wykorzystywanych w dziedzinie detekcji obiektów, można wyróżnić modele otwartoźródłowe:

\begin{itemize}
	\item \textit{Keras RetinaNet} \cite{keras-retinanet-implementation} - implementacja modelu \textit{RetinaNet} \cite{keras-retinanet} oparta na bibliotece \textit{Keras}~\cite{keras};
	\item \textit{YOLO} \cite{yolo-implementation} - implementacja modelu \textit{You Only Look Once} \cite{yolo};
	\item \textit{DeepMask/SharpMask} \cite{deep-sharp-mask} - projekt zawierający implementację modelu \textit{DeepMask} \cite{deepmask} oraz modelu \textit{SharpMask} \cite{sharpmask};
	\item Różne implementacje modelu \textit{Mask R-CNN} \cite{general-mask-rcnn}:
		\begin{itemize}
			\item \textit{Mask R-CNN} \cite{matterport-mask-rcnn} - projekt implementujący model \textit{Mask R-CNN}, noszący taką samą nazwę, oparty na bilbiotekach TensorFlow \cite{tensorflow} i Keras \cite{keras};
			\item \textit{Instance segmentation with OpenCV} \cite{mask-rcnn-opencv} - implementacja modelu \textit{Mask R-CNN} oparta na bibliotece OpenCV \cite{opencv};
			\item \textit{FastMaskRCNN} \cite{fast-mask-rcnn} - implementacja modelu \textit{Mask R-CNN} przy użyciu biblioteki TensorFlow \cite{tensorflow}.
		\end{itemize}
\end{itemize}

Modele \textit{Keras RetinaNet} oraz \textit{YOLO} zostały odrzucone ze względu na fakt, iż pozwalają tylko na detekcję obiektów, nie posiadają funkcjonalności segmentacji instacji.
Model \textit{DeepMask/SharpMask} został odrzucony, ponieważ repozytorium jest zarchiwizowane i~nie rozwijane od 2016 roku, co w przypadku rozwijających się systemów komercyjnych nie jest pożądane.

Z~pozostałych trzech różnych implementacji modelu \textit{Mask R-CNN}, opisanego w \cite{general-mask-rcnn}, wybrano \textit{Mask R-CNN} \cite{matterport-mask-rcnn} z~jednego ważnego powodu - autor udostępnia wagi modelu wytrenowanego na publiczym zbiorze zaanotowanych obrazów \textit{COCO} \cite{coco}. Takie wagi przydatne są do zastosowania uczenia transferowego, opisanego w~Rozdziale \numberref{sec:uczenie-transferowe}.
