% !TeX root = ../../main.tex
\subsection{Sieci neuronowe w kontekście segmentacji instancji}

Do rozwiązania problemu detekcji rozważono kilka modeli open-source:

\begin{itemize}
	\item \textit{Keras RetinaNet} \cite{keras-retinanet} \cite{keras-retinanet-implementation}
	\item \textit{YOLO} \cite{yolo} \cite{yolo-implementation}
	\item \textit{DeepMask/SharpMask} \cite{deepmask} \cite{sharpmask} \cite{deep-sharp-mask}
	\item Różne implementacje modelu \textit{Mask R-CNN} \cite{general-mask-rcnn}
		\begin{itemize}
			\item \textit{Mask\_RCNN} \cite{matterport-mask-rcnn}
			\item \textit{Instance segmentation with OpenCV} \cite{mask-rcnn-opencv}
			\item \textit{FastMaskRCNN} \cite{fast-mask-rcnn}
		\end{itemize}
\end{itemize}

Modele \textit{Keras RetinaNet} oraz \textit{YOLO} zostały odrzucone ze względu na fakt, iż pozwalają tylko na detekcję obiektów, nie posiadają funkcjonalności segmentacji instacji. \\

Model \textit{DeepMask/SharpMask} został odrzucony, ponieważ repozytorium jest zarchiwizowane i nie rozwijane od 2016 roku.

Z~pozostałych trzech różnych implementacji bazujących na pracy \textit{Mask R-CNN} \cite{general-mask-rcnn}, wybór padł na \textit{Mask\_RCNN} \cite{matterport-mask-rcnn} z~jednego, ważnego powodu - autor udostępnia wagi modelu wytrenowanego na publiczym zbiorze zaanotowanych obrazów \textit{COCO} \cite{coco}.
Przydatność tych wag jest opisana w dalszej części pracy \patrz{sec:uczenie-transferowe}.
