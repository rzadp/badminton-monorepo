% !TeX root = ../../main.tex
\section{Wyzwania w problemie segmentacji instancji}

W rozwiązaniu zagadnienia segmentacji instancji kortów przy użyciu sieci neuronowych problem stanowią: ekspozycja kortu, zbiór treningowy oraz czas treningu, które szczegółowo wyjaśniono poniżej.

\subsection*{Ekspozycja kortu}

Podczas gry, kort nigdy nie jest wyeksponowany całkowicie - przysłaniają go gracze, słupki, siatka, sędzia wraz z~jego stanowiskiem, lotka w ruchu.
Dodatkowo w przypadku użycia klatek z~transmisji telewizyjnych, często w postprodukcji na obszar kortu nałożona jest reklama czy logo organizatora meczu,  tak więc rozwiązanie musi być odporne na takie artefakty.

\subsection*{Zbiór treningowy}

Wytrenowanie sieci neuronowej dla złożonych zadań wymaga dużej liczby rekordów w zbiorze treningowym, aby uzyskać satysfakcjonujące wyniki.
Trening sieci wykrywającej obszar kortu prawdopodobnie wymagałby setek lub tysięcy obrazów treningowych.
Przygotowanie tak licznego zbioru jest problematyczne ze względu na to, że:

\begin{itemize}
	\item brakuje ogólnodostępnych baz z obrazami kortów przygotowanych na potrzeby treningu sieci neruonowych;
	\item firma posiada dane z jednej lokalizacji, a to jest niewystarczające, ponieważ lokalizacje różnią się miedzy sobą np. odległością między kortami, kolorem podłogi i kolorem samego kortu, siatką czy wyglądem stanowiska sędziowskiego.
\end{itemize}

\subsection*{Czas treningu}

Trening dużej sieci neuronowej przy użyciu jednej karty GPU potrafi trwać dziesiątki godzin.
Jest to duża przeszkoda w eksprymentach z~różnymi zbiorami treningowymi, hiperparametrami czy zmianami architektury sieci.
