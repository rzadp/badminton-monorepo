% !TeX root = ../../main.tex
\chapter{Podsumowanie}

Głównym celem niniejszej pracy była eksploracja zagadnienia segmentacji instancji kortu do badmintona w obrazie wejściowym, skupiając się na rozwiązaniach z użyciem sieci neuronowych, tak aby można było porównać takie rozwiązania z rozwiązaniem algorytmicznym wykorzystywanym przez firmę \blue{} na potrzeby systemu automatycznego sędziowania.

Wytrenowanie sieci neuronowej wymaga zbioru treningowego - niestety z powodu braku ogólnodostępnych baz z takimi danymi, w ramach pracy konieczne było stworzenie własnych zbiorów danych. Podział tych zbiorów na podzbiory treningowe, walidacyjne i testowe został dobrany po eksperymentach mających na celu zmaksymalizowanie skuteczności treningu. Na elementy skonstruowanych zbiorów składają się obrazy pochodzące z Internetu, obrazy dostarczone przez firmę \blue{}, oraz obrazy sztucznie wygenerowane przez stworzony w ramach niniejszej pracy generator. Takie sztucznie stworzone obrazy nie wyglądają realistycznie, dlatego też przebadano ich wpływ na skuteczność treningu. Testy wykazały, że dodanie do zbioru treningowego takich obrazów pozytywnie wpływa na wynik trenowanej na niej sieci w przypadku mało zróżnicowanego zbioru, to znaczy takiego w którym wszystkie rekordy są podobne do siebie. W przypadku bardziej zróżnicowanego zbioru, dodanie sztucznych rekordów nie polepszyło wyników sieci.

W kolejnych krokach przebadano modyfikację implementacji sieci \textit{Mask R-CNN} \cite{matterport-mask-rcnn}, mającą na celu poprawienie dokładności trenowanej sieci, mierzonej jako procent poprawnie zaklasyfikowanych pikseli w obrazie. Eksperymenty wskazują na to, że dzięki wprowadzonej modyfikacji osiągnięta została większa skuteczność sieci, testowana na skontruowanych zbiorach. Tak wytrenowana sieć, ze skutecznością na poziomie 97\% poprawnie zaklasyfikowanych pikseli w obrazie, może konkurować z algorytmicznym podejściem firmy \blue{}. Prawdopodobnie najlepsze wyniki mogłyby być osiągnięte poprzez połączenie tych dwóch rozwiązań - jednak to wykracza poza zakres niniejszej pracy.

Za jeden z pobocznych celów pracy obrano eksperymenty dotyczące rozszerzenia zagadnienia na inne zastosowania, na przykład na sporty inne niż badminton. Niestety czasochłonność zarówno konstruowania zbiorów danych jak i przeprowadzania treningów sieci sprawiła, iż z tego celu pobocznego zrezygnowano w ramach tej pracy.
Nawet bez próby rozszerzenia zagadnienia na inne zastosowania, czas przeprowadzania eksperymentów był problematyczny. Prawdopodobnie można to poprawić poprzez jeszcze agresywniejsze zastosowanie uczenia transferowego. W pracy uczenie transferowe jest wykorzystywane w przypadku trenowania sieci na skonstruowanych zbiorach, wykorzystując wagi wytrenowane na innym, niezwiązanym z tematyką badmintona zbiorze. Pomysł na usprawnienie przeprowadzania eksperymentów polega na skonstruowaniu kolejnego, odpowiednio małego zbioru danych i wstępnym wytrenowaniu na nim modelu, którego wagi stanowiłyby bazę dla wszystkich eksperymentów. Takie podejście mogłoby skrócić czas potrzebny na przeprowadzenie treningów.
