% !TeX root = ../../main.tex
\chapter{Wstęp}

Niniejsza praca dotyczy problemu wykrywania obszaru kortu do badmintona na obrazie z~wykorzystaniem sieci neuronowych.
Koncentracja na kortach do badmintona (a~nie kortach do gry w ogóle) wynika ze współpracy z~firmą \blue{}, która zajmuje się systemami sędziowania meczy badmintona.

\section{Motywacja}

Temat pracy został zaproponowany przez firmę \blue{}, która zajmuje się sędziowaniem meczów badmintona i pracuje nad automatyzacją tego zagadnienia.
Wymaga to między innymi śledzenia toru lotu lotki, śledzenia zawodników, wykrycia kortu.
Konkretne piksele kortu potrzebne są do rozstrzygania wątpliwości w kwestii wyjścia lotki poza obszar boiska.

W polu widzenia kamery często znajduje się więcej niż jeden kort, stąd semantyczna segmentacja nie jest wystarczająca i wymagana jest segmentacja instancji~\footnote{\myref{sec:typy_detekcji}}.

\section{Cel pracy}

Celem głównym pracy jest zbadanie użyteczności sieci neuronowych na potrzeby komercyjnego systemu sędziowania meczy badmintona.
Aktualne podejście firmy \blue{} do problemu wykrywania kortu polega na algorytmicznym wykrywaniu linii kortu, bez użycia sieci neuronowych.
Praca stanowi krok w eksploracji tematu wykrywania kortu z~użyciem sieci neuronowych i daje podstawy do decyzji, czy podejście to nadaje się do komercyjnego zastosowania (zamiast, lub w połączeniu z obecnym, algorytmicznym rozwiązaniem).

Celem pośrednim, wynikającym z~celu głównego jest analiza istniejących rozwiązań w literaturze, rozpatrując zarówno podejścia wykorzystujące sieci neuronowe oraz rozwiązania klasyczne, to znaczy bez wykorzystania sieci.
\TODO{Wymienić rozwiązania konkretnie, podsumowując \myrefx{sec:metody_detekcji}}

Dodatkowe cele pośrednie, również wynikające z~celu głównego, stanowią stworzenie sztucznego zbioru treningowego oraz optymalizacja wykorzystywanej sieci, zarówno pod kątem czasu treningu jak i wymagań pamięciowych karty graficznej.

Kolejnym celem pobocznym jest rozszerzenie zagadnienia na inne zastosowania, aby praca była potencjalnie użyteczna dla większego grona odbiorców niż tylko osób z~branży badmintona.
\\

\section{Wkład własny}

Wkład własny autora został zebrany w następujących punktach:

\begin{itemize}
	\item Zbiór danych
    \begin{itemize}
      \item Skonstruowanie dwóch zbiorów danych \patrz{sec:zbiory}
      \item Zbadanie podziałów zbiorów na skuteczność sieci \patrz{sec:podzial}
			\item Stworzenie generatora sztucznych danych \patrz{sec:generator}
			\item Zbadanie wpływu sztucznych danych na skuteczność treningu \patrz{sec:generator}
		\end{itemize}
	\item Sieć neuronowa
	\begin{itemize}
		\item Zebranie głównych idei kolejnych iteracji wybranej sieci neuronowej \patrz{sec:architekrura_mask_rcnn}
		\item Modyfikacja i przygotowanie sieci na potrzeby analizowanego zagadnienia \patrz{sec:zaproponowana_architektura}
	\end{itemize}
\end{itemize}
