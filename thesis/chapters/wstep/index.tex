% !TeX root = ../../main.tex
\chapter{Wstęp}

Niniejsza praca dotyczy problemu detekcji obszaru kortu do badmintona w obrazie z~wykorzystaniem sieci neuronowych.
Zagadnienie detekcji obszaru kortu do badmintona wynika ze współpracy z~firmą \blue{}, która zajmuje się tworzeniem systemu sędziowania meczów badmintona.

\section{Motywacja}

Temat pracy został zaproponowany w ramach współpracy z firmą \blue{}, która zajmuje się tworzeniem systemu do sędziowania meczów badmintona i obecnie pracuje nad automatyzacją śledzenia toru lotu lotki, śledzenia zawodników oraz detekcji obszaru kortu.
Detekcja obszaru kortu potrzebna jest do rozstrzygania wątpliwości w kwestii wyjścia lotki poza obszar kortu.

W polu widzenia kamery często znajduje się więcej niż jeden kort, stąd semantyczna segmentacja nie jest wystarczająca i wymagana jest segmentacja instancji \myrozdzial{sec:typy_detekcji}.

\section{Cel pracy}

Celem głównym pracy było zbadanie użyteczności sieci neuronowych na potrzeby komercyjnego systemu sędziowania meczy badmintona.
Aktualne podejście firmy \blue{} polega na algorytmicznej detekcji obszaru kortu, bez użycia sieci neuronowych \myrozdzial{sec:aglorytmiczna_detekcja}.
Praca stanowi krok w eksploracji tematu detekcji obszaru kortu z~użyciem sieci neuronowych i daje podstawy do decyzji, czy podejście to nadaje się do komercyjnego zastosowania zamiast lub w połączeniu z obecnym algorytmicznym rozwiązaniem.

Celem pośrednim, wynikającym z powyższego celu głównego, jest analiza istniejących rozwiązań w literaturze, zarówno wykorzystujących sieci neuronowe, jak i rozwiązań klasycznych (bez wykorzystania sieci neuronowych).

Dodatkowo celem była optymalizacja wykorzystywanej architektury sieci oraz konstrukcja zbiorów danych rozszerzonych o sztucznie wygenerowane obrazy.
\\

\section{Wkład własny}

Wkład własny autora niniejszej pracy został zebrany w następujących punktach:

\newcommand{\zaprojektowanie}[2]{zaprojektowanie i przeprowadzenie testów detekcji obszaru kortu z wykorzystaniem zbioru \textit{#1} \myrozdzial{#2}}

\begin{itemize}
	\item zbiór danych:
    \begin{enumerate}
			\item stworzenie generatora sztucznych danych w celu zwiększenia ilości rekordów treningowych \myrozdzial{sec:generator},
			\item zbadanie wpływu użycia sztucznych danych treningowych na skuteczność treningu \myrozdzial{sec:eksperymenty_wyniki},
      \item skonstruowanie dwóch zbiorów danych - zbioru \text{high} i zbioru \textit{low} \myrozdzial{sec:zbiory},
      \item zbadanie wpływu podziału zbiorów na skuteczność metody (patrz Rozdział \numberref{sec:podzial} oraz Rozdział~\numberref{sec:podzial_eksperyment});
		\end{enumerate}
	\item architektura sieci neuronowej:
	\begin{itemize}
		\item przedstawienie kolejnych iteracji wybranej sieci neuronowej \myrozdzial{sec:architekrura_mask_rcnn},
		\item modyfikacja architektury sieci na potrzeby analizowanego zagadnienia \myrozdzial{sec:zaproponowana_architektura},
		\item \zaprojektowanie{low}{sec:experymenty_low},
		\item \zaprojektowanie{high}{sec:experymenty_high},
		\item przeprowadzenie integracji zaproponowanej metody z systemem firmy \blue{} (Rozdział \numberref{sec:integrationblue}).
	\end{itemize}
\end{itemize}
