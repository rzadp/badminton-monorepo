% !TeX root = ../../main.tex
\chapter{Wstęp}

Niniejsza praca dotyczy problemu detekcji obszaru kortu do badmintona na obrazie z~wykorzystaniem sieci neuronowych.
Zagadnienie detekcji obszaru kortu do badmintona wynika ze współpracy z~firmą \blue{}, która zajmuje się tworzeniem systemu sędziowania meczów badmintona.

\section{Motywacja}

Temat pracy został zaproponowany w ramach współpracy z firmą \blue{}, która zajmuje się tworzeniem systemu do sędziowania meczów badmintona i obecnie pracuje nad automatyzacją śledzenia toru lotu lotki, śledzenia zawodników oraz detekcji obszaru kortu.
Detekcja obszaru kortu potrzebna jest do rozstrzygania wątpliwości w kwestii wyjścia lotki poza obszar kortu.

W polu widzenia kamery często znajduje się więcej niż jeden kort, stąd semantyczna segmentacja nie jest wystarczająca i wymagana jest segmentacja instancji \patrzx{sec:typy_detekcji}.

\section{Cel pracy}

Celem głównym pracy było zbadanie użyteczności sieci neuronowych na potrzeby komercyjnego systemu sędziowania meczy badmintona.
Aktualne podejście firmy \blue{} polega na algorytmicznej detekcji obszaru kortu, bez użycia sieci neuronowych \patrzx{sec:aglorytmiczna_detekcja}.
Praca stanowi krok w eksploracji tematu detekcji obszaru kortu z~użyciem sieci neuronowych i daje podstawy do decyzji, czy podejście to nadaje się do komercyjnego zastosowania zamiast lub w połączeniu z obecnym algorytmicznym rozwiązaniem.

Celem pośrednim, wynikającym z powyższego celu głównego, jest analiza istniejących rozwiązań w literaturze, zarówno wykorzystujących sieci neuronowe, jak i rozwiązań klasycznych (bez wykorzystania sieci neuronowych).

Dodatkowo celem była optymalizacja wykorzystywanej architektury sieci oraz konstrukcja zbiorów danych wzbogaconych o sztucznie wygenerowane obrazy.
\\

\section{Wkład własny}

Wkład własny autora niniejszej pracy został zebrany w następujących punktach:

\begin{itemize}
	\item Zbiór danych:
    \begin{enumerate}
      \item Skonstruowanie dwóch zbiorów danych \patrz{sec:zbiory}
      \item Zbadanie wpływu podziałów zbiorów na skuteczność sieci \patrz{sec:podzial}
			\item Stworzenie generatora sztucznych danych \patrz{sec:generator}
			\item Zbadanie wpływu sztucznych danych na skuteczność treningu \patrz{sec:generator}
		\end{enumerate}
	\item Sieć neuronowa
	\begin{itemize}
		\item Przedstawienie głównych idei kolejnych iteracji wybranej sieci neuronowej \patrz{sec:architekrura_mask_rcnn}
		\item Modyfikacja i przygotowanie sieci na potrzeby analizowanego zagadnienia \patrz{sec:zaproponowana_architektura}
	\end{itemize}
\end{itemize}
