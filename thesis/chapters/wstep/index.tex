% !TeX root = ../../main.tex
\chapter{Wstęp}

Niniejsza praca dotyczy problemu wykrywania obszaru kortu do badmintona z~wykorzystaniem sieci neuronowych.
Koncentracja na kortach do badmintona (a~nie kortach do gry w ogóle) wynika ze współpracy z~firmą BLUE, która zajmuje się systemami sędziowania meczy badmintona.
Praca zawiera jednak wnioski na temat generalizacji problemu na inne sporty.

\section{Motywacja}

Temat pracy został zaproponowany przez firmę BLUE, która zajmuje się sędziowaniem meczy badmintona i pracuje nad automatyzacją tego zagadnienia.
Wymaga to między innymi śledzenia toru lotu lotki, śledzenia zawodników, wykrycia kortu.
Konkretne piksele kortu potrzebne są do rozstrzygania wątpliwości w kwestii autu.

W polu widzenia kamery często znajduje się więcej niż jeden kort, stąd semantyczna segmentacja nie jest wystarczająca i wymagana jest segmentacja instancji\footnote{\myref{sec:typy_detekcji}}.

\section{Cel pracy}

Celem głównym pracy jest zbadanie użyteczności sieci neuronowych na potrzeby komercyjnego systemu sędziowania meczy badmintona.
Aktualne podejście firmy BLUE do problemu wykrywania kortu polega na algorytmicznym wykrywaniu linii kortu, bez użycia sieci neuronowych.
Praca stanowi krok w eksploracji tematu wykrywania kortu z~użyciem sieci neuronowych i daje podstawy do decyzji, czy podejście to nadaje się do komercyjnego zastosowania.

Celem pośrednim, wynikającym z~celu głównego jest analiza istniejących rozwiązań w literaturze, rozpatrując zarówno podejścia wykorzystujące sieci neuronowe oraz rozwiązania klasyczne, to znaczy bez wykorzystania sieci.

Dodatkowe cele pośrednie, również wynikające z~celu głównego stanowią stworzenie sztucznego zbioru treningowego oraz optymalizacja wykorzystywanej sieci, zarówno pod kątem czasu treningu jak i wymagań pamięciowych karty graficznej.

Cel poboczny stanowi rozszerzenie zagadnienia na inne zastosowania, aby praca była potencjalnie użyteczna dla większego grona odbiorców niż tylko osób z~branży badmintona.
\\

\section{Wkład własny}

\begin{itemize}
	\item Zbiór treningowy
		\begin{itemize}
			\item Stworzenie generatora sztucznych danych \patrz{sec:generator}
			\item Zbadanie wpływu sztucznych danych na skuteczność treningu \patrz{sec:wyniki_generator}
		\end{itemize}
	\item Sieć neuronowa
	\begin{itemize}
		\item Zebranie głównych idei kolejnych iteracji wybranej sieci neuronowej \patrz{sec:architekrura_mask_rcnn}
		\item Modyfikacja i przygotowanie sieci na potrzeby tytułowego problemu \patrz{sec:zaproponowana_architektura}
	\end{itemize}
	\item Zbadanie kwestii generalizacji problemu \patrz{sec:generalizacja}
\end{itemize}
