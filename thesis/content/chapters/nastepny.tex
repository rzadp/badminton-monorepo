\chapter{Następny rozdział}

Lorem ipsum dolor sit amet, consetetur sadipscing elit, sed diam nonumyeirmod tempor invidunt ut labore et dolore magna aliquyam erat, sed diamvoluptua. At vero eos et accusam et justo duo dolores et ea rebum. Stet clita kasd gubergren, no sea takimata sanctus est Lorem ipsum dolor sit amet.Lorem ipsum dolor sit amet, consetetur sadipscing elitr, sed diam nonumyeirmod tempor invidunt ut labore et dolore magna aliquyam erat, sed diamvoluptua. At vero eos et accusam et justo duo dolores et ea rebum. Stet clita kasd gubergren, no sea takimata sanctus est Lorem ipsum dolor sit amet.


\section{Macierze}

Prosta macierz:
\[
\begin{matrix}
a & b & c & d \\
d & e & f & g \\
1 & 1 & 1 & 1
\end{matrix}
\]
Macierz z nawiasami okrągłymi:
\[
A = 
\begin{pmatrix}
a & b & c & d \\
d & e & f & g \\
1 & 1 & 1 & 1
\end{pmatrix}
\]
Macierz z nawiasami kwadratowymi:
\[
\begin{bmatrix}
a & b & c & d \\
d & e & f & g \\
1 & 1 & 1 & 1
\end{bmatrix}
\]
Można też ogólniejsze środowisko:
\[
\renewcommand{\arraystretch}{0.8}
\begin{array}{ccc}
1 & 0 & 0 \\
0 & 1 & 0 \\
0 & 0 & 1 \\
\end{array}
\]
Nawiasy klamrowe:
\[
\left\{
\renewcommand{\arraystretch}{0.8}
\begin{array}{ccc}
1 & 0 & 0 \\
0 & 1 & 0 \\
0 & 0 & 1 \\
\end{array}\right\}
\]

\begin{definition}
Niech $A\neq \emptyset$, $n \in \mathbb{N}$. Każde przekształcenie $f:A^n \rightarrow A$ nazywamy \textit{$n$-arną operacją} lub \textit{działaniem} określonym na $A$.
0-arne operacje to wyróżnione stałe.
\end{definition}


\begin{definition}[Algebra]
Parę uporządkowaną $(A,F)$, gdzie $A\neq \emptyset$ jest zbiorem, a $F$ jest rodziną operacji określonych na $A$, nazywamy \textit{algebrą} (lub \textit{$F$-algebrą}). Zbiór $A$ nazywa się \textit{zbiorem elementów}, \textit{nośnikiem} lub \textit{uniwersum} algebry $(A,F)$, a $F$ \textit{zbiorem operacji elementarnych}.
\end{definition}

\begin{proposition}
Stwierdzam więc ostatnio, że doszedłszy do granicy, pozostaje mi tylko przy tej granicy biwakować albo zawrócić, możliwie też szukać przejścia czy wyjścia na nowe obszary.
\end{proposition}
