\chapter{Wstęp}

Niniejsza praca dotyczy problemu wykrywania obszaru kortu do badmintona w wykorzystaniem sieci neuronowych.
Koncentracja na kortach do badmintona (a nie kortach do gry w ogóle) wynika ze współpracy z firmą BLUE, która zajmuje się systemami sędziowania meczy badmintona. Praca zawiera jednak wnioski na temat generalizacji problemu na inne sporty.

\section{Cel pracy}

Celem głównym pracy jest zbadanie użyteczności sieci neuronowych na potrzeby komercyjnego systemu sędziowania meczy badmintona. Aktualne podejście firmy BLUE do problemu wykrywania kortu polega na algorytmicznym wykrywaniu linii kortu, bez użycia sieci neuronowych. Praca stanowi krok w eksploracji tematu wykrywania kortu z użyciem sieci neuronowych i daje podstawy do decyzji, czy podejście to nadaje się do komercyjnego zastosowania.

Cel poboczny stanowi wzbogacenie treści o temat generalizacji problemu na inne sporty, aby praca była potencjalnie użyteczna dla większego grona czytelników niż tylko osób z branży badmintona.

\section{Wkład własny}

\begin{itemize}
	\item Zbiór treningowy
		\begin{itemize}
			\item Stworzenie generatora sztucznych danych
			\item Zbadanie wpływu sztucznych danych na skuteczność treningu
		\end{itemize}
	\item Sieć neuronowa
	\begin{itemize}
		\item Zebranie głównych idei kolejnych iteracji wybranej sieci neuronowej
		\item Modyfikacja i przygotowanie sieci na potrzeby tytułowego problemu
	\end{itemize}
	\item Zbadanie kwestii generalizacji problemu
\end{itemize}
