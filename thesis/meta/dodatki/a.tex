% !TeX root = ../../main.tex
\chapter*{Dodatek A - Instrukcja instalacji}
\rhead{\uppercase{Dodatek A - Instrukcja instalacji}}
\addcontentsline{toc}{chapter}{\currentname}
\label{sec:instrukcja-instalacji}

Aplikacja służąca do konstrukcji zbiorów danych, trenowania sieci neuronowej oraz detekcji obszaru kortu składa się z~następujących elementów: generatora sztucznych danych, anotatora danych oraz zmodyfikowananej implementacji sieci \textit{Mask R-CNN} \cite{matterport-mask-rcnn} (wraz ze skryptami służącymi do trenowania i obliczania skuteczności wytrenowanej sieci).

\section*{Generator sztucznych danych}

Wymagane zależności systemowe:

\begin{itemize}
  \item \textit{Node.js} \cite{nodejs};
  \item \textit{npm CLI} \cite{npm}.
\end{itemize}

Instalacja wymaganych pakietów:

\begin{verbatim}
  $ npm install
\end{verbatim}

Uruchomienie generatora lokalnie (pod adresem \textit{http://localhost:1234}):

\begin{verbatim}
  $ npm run dev
\end{verbatim}

Zbudowanie aplikacji, którą można wrzucić na serwer jako statyczna strona internetowa:

\begin{verbatim}
  $ npm run build
\end{verbatim}

\section*{Anotator danych}
Anotator danych jest statyczną stroną \textit{HTML} z zawartym kodem \textit{JavaScript} i stylami \textit{CSS} w~jednym pliku \textit{index.html}, dzięki czemu nie wymaga instalacji. Do uruchomienia anotatora wymagana jest jedynie przeglądarka internetowa, a w~przypadku pliku \textit{index.html} na lokalnym dysku nie jest wymagane połączenie z Internetem.

\section*{Zmodyfikowana implementacja sieci \textit{Mask R-CNN}}

Wymagane zależności systemowe:

\begin{itemize}
  \item \textit{Python} \cite{python} w wersji 3.5
  \item \textit{pip3} \cite{pip}
  \item karta graficzna - opcjonalnie, ponieważ program zadziała też na CPU, lecz trening na~samym CPU trwa nawet kilkadziesiąt razy dłużej, przez co praktycznie zastosowanie nie jest możliwe.
\end{itemize}

Instalacja wymaganych pakietów:

\begin{verbatim}
  $ pip3 install \
  numpy==1.15.4 \
  scipy==1.2.0 \
  cython==0.29.2 \
  matplotlib==3.0.2 \
  scikit-image==0.14.1 \
  tensorflow==1.12.0 \
  keras==2.2.4 \
  h5py==2.9.0 \
  imgaug==0.2.7 \
  IPython[all]==7.2.0
\end{verbatim}