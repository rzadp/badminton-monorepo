% !TeX root = ../main.tex
\begin{thebibliography}{20}%jak ktoś ma więcej książek, to niech wpisze większą liczbę
  \addcontentsline{toc}{chapter}{Bibliografia}
% \bibitem[numerek]{referencja} Autor, \emph{Tytuł}, Wydawnictwo, rok, strony
% cytowanie: \cite{referencja1, referencja 2,...}

%\bibitem[1]{Ktos} A. Author, \emph{Title of a book}, Publisher, year, page--page.
%\bibitem[2]{Innyktos} J. Bobkowski, S. Dobkowski, Jak stworzyć bibliografię w BibTeX-u, \emph{Czasopismo nr}, rok, strona--strona.
%\bibitem[3]{B} C. Brink, Power structures, \emph{Algebra Universalis 30(2)}, 1993, 177--216.
%\bibitem[4]{H} F. Burris, H. P. Sankappanavar, \emph{A Course of Universal Algebra}, Springer-Verlag, Nowy Jork, 1981.


\bibitem[1]{keras-retinanet} https://github.com/fizyr/keras-retinanet
\bibitem[2]{yolo} https://pjreddie.com/darknet/yolo
\bibitem[3]{deep-sharp-mask} https://github.com/facebookresearch/deepmask
\bibitem[4]{general-mask-rcnn} https://arxiv.org/abs/1703.06870
\bibitem[5]{matterport-mask-rcnn} https://github.com/matterport/Mask\_RCNN
\bibitem[6]{mask-rcnn-opencv} https://www.pyimagesearch.com/2018/11/26/instance-segmentation-with-opencv
\bibitem[7]{fast-mask-rcnn} https://github.com/CharlesShang/FastMaskRCNN
\bibitem[8]{vgg-via} http://www.robots.ox.ac.uk/~vgg/software/via/
\bibitem[9]{rcnn} CITATION-NEEDED
\bibitem[10]{selective-search} http://www.huppelen.nl/publications/selectiveSearchDraft.pdf
\bibitem[11]{fast-rcnn} CITATION-NEEDED
\bibitem[12]{region-of-interest-pooling} CITATION-NEEDED
% \bibitem[13]{} CITATION-NEEDED
% \bibitem[14]{} CITATION-NEEDED
% \bibitem[15]{} CITATION-NEEDED

\end{thebibliography}

\TODO{todo}
