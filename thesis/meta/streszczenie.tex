\begin{verbatim}
  streszcznie - rozwinąć o 5 zdań (o motywacji, dlaczego to trudne, wiecej 
szczegółów o sieci , na czym polegały eksperymenty, jaki uzyskano 
najlepszy wynik)
\end{verbatim}

Niniejsza praca dotyczy problemu wykrywania kortu do badmintona przy użyciu sieci neuronowych. Przedstawia sposób działania użytej sieci neuronowej oraz jej modyfikacje wraz z historycznym kontekstem rozwoju sieci dla lepszego zrozumienia. 

W pracy poruszono problem przygotowania zbioru treningowego z użyciem generatora sztucznych danych, a także zagadnienie generalizacji problemu na inne sporty.

Dodatkowo opisano aplikację wraz z rezultatami eksperymentów, które zostały przeprowadzone. Eksperymenty dotyczą wpływu sztucznych danych na trening sieci oraz skuteczności generalizacji problemu. \\

\noindent \textbf{Słowa kluczowe:} sieci neuronowe, sieci konwolucyjne, rozpoznawanie obrazów
