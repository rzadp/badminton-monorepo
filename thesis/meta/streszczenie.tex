% !TeX root = ../main.tex
Niniejsza praca dotyczy problemu wykrywania obszaru kortu do badmintona przy użyciu sieci neuronowych.
Przedstawia sposób działania użytej sieci neuronowej \textit{Mask R-CNN} \cite{general-mask-rcnn} oraz jej modyfikacje wraz z~historycznym kontekstem rozwoju sieci dla lepszego zrozumienia jej działania.

Temat pracy został zaproponowany przez firmę ``BLUE'' działającej w branży badmintona - stąd też ukierunkowanie na wykrywanie konkretnie kortów badmintona.

Główne problemy poruszone w pracy to trening sieci bez odpowiednio licznego zbioru treningowego, minimalizacja czasu treningu, ograniczenie wymagań pamięciowych GPU oraz dokładność z~jaką wykrywane są piksele obszaru kortu.

Dołączona do pracy aplikacja składa się ze zmodyfikowanej sieci \textit{Mask R-CNN}, generatora sztucznych danych treningowych, narzędzia do anotacji danych treningowych oraz instrukcji ukierunkowanej na umożliwienie wiarygodnego zreprodukowania eksperymentów.

Przeprowadzone eksperymenty dotyczą porównania wyników oryginalnej i zmodyfikowanej sieci, skuteczności treningu bez i z~użyciem generatora sztucznych danych, a~także tematu generalizacji zagadnienia na inne sporty.
\\

\TODO{Najlepsze wyniki / Wyniki wskazują na (...); (Uzupełnić podsumowując \myrefx{sec:wyniki_zmodyfikowanej}}

\noindent \textbf{Słowa kluczowe:} sieci neuronowe, sieci konwolucyjne, rozpoznawanie obrazów
