% !TeX root = ../main.tex
Niniejsza praca dotyczy problemu wykrywania obszaru kortu do badmintona na obrazie, przy użyciu sieci neuronowych.
Przedstawia sposób działania użytej sieci neuronowej \textit{Mask R-CNN}~\cite{general-mask-rcnn} oraz jej modyfikacje wraz z~historycznym kontekstem rozwoju sieci dla lepszego zrozumienia jej działania.
Temat pracy został zaproponowany przez firmę \blue{} działającej w branży badmintona - stąd też ukierunkowanie na wykrywanie konkretnie kortów badmintona.
Główne problemy poruszone w pracy to skonstruowanie zbioru danych na potrzeby treningu sieci oraz dokładność z~jaką wykrywane są piksele obszaru kortu.

Dołączona do pracy aplikacja składa się ze zmodyfikowanej sieci \textit{Mask R-CNN}, generatora sztucznych danych treningowych, narzędzia do anotacji danych treningowych oraz instrukcji ukierunkowanej na umożliwienie zreprodukowania eksperymentów.
Przeprowadzone eksperymenty dotyczą porównania wyników oryginalnej i zmodyfikowanej sieci, a~także skuteczności treningu bez i z~użyciem generatora sztucznych danych.
Najlepsze osiągnięte wyniki wynoszą średnio 97\% poprawnie wykrytych pikseli obszaru kortu na obrazie.
\\

\noindent \textbf{Słowa kluczowe:} sieci neuronowe, sieci konwolucyjne, rozpoznawanie obrazów, segmentacja~instancji
